% \section{Main result}
% % Main result \cite{kalabic_14acc,murray_book,Bhat2000}.

% 数学的準備や主結果などのメイン部分 \cite{latex}.
\section{Preliminaries}
\textcolor{red}{In this section, we derive the two-wheeled drone model and ECBF.}
\subsection{Two-wheeled drone model}
\begin{figure}[t]
    \centering
    \includegraphics[width=1\linewidth]{./drawing/pdf/Two-Wheeled_Drone.pdf}
    \caption{Two-wheeled drone moving on a wall.}
    \label{fig:two-wheeled_drone_moving_on_a_wall}
\end{figure}

Consider a two-wheeled drone moving on a wall illustrated in Fig.~\ref{fig:two-wheeled_drone_moving_on_a_wall}.
We choose an inertial frame $e_i \in \mathbb{R}^3$, $i \in \{1, 2, 3 \}$ and a body-fixed frame $\{b_x, b_y, b_z\}$.
The origin of the body-fixed frame is the center of mass of the drone.
\textcolor{red}{
The mass of the drone is $m \in \mathbb{R}$, acceleration of gravity is $g \in \mathbb{R}$, position in the world frame is $p = [x ~ y ~ z]^T \in \mathbb{R}^3$, body velocity is $v^b = [v_1^b ~ v_2^b ~ v_3^b]^T \in \mathbb{R}^ 3$, thrust is $f \in \mathbb{R}$, inertia tensor is $J \in \mathbb{R}^{3 \times 3}$, $ZXY$ Euler angles is $\eta = [\alpha ~ \beta ~ \gamma]^T \in \mathbb{R}^3$, rotation matrix is $R \in SO(3)$, body angular velocity is $\omega^b = [\omega_1^b ~ \omega_2^b ~ \omega_3^b]^T \in \mathbb{R}^3$, body torque is $\tau^b \in \mathbb{R}^3$, and $ \lambda_j $, $ j \in \{1,2,3\} $ is Lagrangian multipliers.
}
\textcolor{red}{
The two-wheeled drone running on the wall as shown in Fig.~\ref{fig:two-wheeled_drone_moving_on_a_wall} has the following constraints,
\begin{enumerate}
    \item Not to slip on the wall.
    \item Constant distance from the wall.
    \item Not to rotate around the wall.
\end{enumerate}
These constraints can be written as the following equation,
\begin{align}
    \begin{cases}
        \label{eq:constraints_of_the_two-wheeled_drone}
        f_1 &= v_2^b = r_{12} \dot{x} + r_{22} \dot{y} + r_{32} \dot{z} = 0\\
        f_2 &= x + \frac{D}{2} - L = 0\\
        f_3 &= e_1^T b_y = r_{12} = e_1^T (\eta - \eta_0)
    \end{cases}
\end{align}
where $ D \in \mathbb{R} $ is the dismeter of the wheel, $ L \in \mathbb{R} $ is the distance from the origin to the wall.
From the third equation of \eqref{eq:constraints_of_the_two-wheeled_drone}, the first equation of \eqref{eq:constraints_of_the_two-wheeled_drone} can be rewritten as follows,
\begin{align}
    \label{eq:first_constraint_rewritten}
    f_1 = r_{22} \dot{y} + r_{32} \dot{z} = A \dot{p} = 0
\end{align}
where $ A = [0 ~ r_{22} ~ r_{32}] $.
}

The kinematics of the translational motion is given by
\begin{align}
    \label{eq:kinematics_of_translation}
    \dot{p} &= Rv^b = 
    \begin{bmatrix}
        0\\
        - \frac{v_3^b \sin \beta}{\cos \gamma}\\
        \frac{v_3^b \cos \beta}{\cos \gamma}
    \end{bmatrix}
    = \frac{1}{\cos \gamma} R_2(\beta) (v_3^b e_3)
\end{align}
where
\begin{align}
    R &=
    \begin{bmatrix}
        r_{11} & r_{12} & r_{13} \\
        r_{21} & r_{22} & r_{23} \\
        r_{31} & r_{32} & r_{33}
    \end{bmatrix}\\
    &=
    \begin{bmatrix}
        \cos \gamma & 0 & \sin \gamma \\
        \sin \beta \sin \gamma & \cos \beta & - \sin \beta \cos \gamma \\
        - \cos \beta \sin \gamma & \sin \beta & \cos \beta \cos \gamma
    \end{bmatrix}
    \label{eq:rotation_matrix}
\end{align}

\begin{align}
    \label{eq:definition_of_R_2}
    R_2(\beta) =
    \begin{bmatrix}
        1 & 0 & 0 \\
        0 & \cos \beta & - \sin \beta \\
        0 & \sin \beta & \cos \beta
    \end{bmatrix}.
\end{align}

The kinematics of the rotational motion in the $ZXY$ Euler angles representation is given by
\begin{align}
    \label{eq:kinematic_of_rotation_with_constraint_simplified_ver}
\dot{\eta} = \Phi(\eta) \omega^b
\end{align}
where
\begin{align*}
    \Phi(\eta) = 
    \begin{bmatrix}
        \frac{\sin \gamma}{\cos \beta} & 0 & \frac{\cos \gamma}{\cos \beta}\\
        \cos \gamma & 0 & - \sin \gamma\\
        \frac{\sin \beta \sin \gamma}{\cos \beta} & 1 & \frac{\sin \beta \cos \gamma}{\cos \beta}
    \end{bmatrix}.
\end{align*}
The inverse of $ \Phi(\eta) $ is $ \Psi = \Phi^{-1} $, which is given by
\begin{align}
    \label{eq:definition_of_Psi}
    \Psi(\eta) =
    \begin{bmatrix}
        \cos \beta \sin \gamma & \cos \gamma & 0 \\
        - \sin \beta & 0 & 1 \\
        \cos \beta \cos \gamma & - \sin \gamma & 0
    \end{bmatrix}.
\end{align}

The dynamics of the translational motion is given by
\begin{align}
	\label{eq:translational_dynamics}
	m\ddot{p}+mge_3 +\lambda_1 A^T +\lambda_2 e_1 = Rfe_3
\end{align}
\textcolor{red}{where $ \lambda_1 $ is the term about side slipping, $ \lambda_2 $ is the term about the force to push the drone against the wall, and
\begin{align}
    A = [0 ~ r_{22} ~ r_{32}].
\end{align}}

The dynamics of the rotational motion \cite{nonami_autonomous_2010} is given by
\begin{align}
    \label{eq:rotational_el_eq}
    M(\eta) \ddot{\eta} + C(\eta, \dot{\eta}) \dot{\eta} = \Psi(\eta)^T \tau^b - \lambda_3 e_1
\end{align}
\textcolor{red}{where $ \lambda_3 $ is the term about yawing and
\begin{align}
    \label{eq:definition_of_rotational_motion}
    M(\eta) &= \Psi(\eta)^T J \Psi(\eta) \\
    C(\eta,\dot{\eta}) &= \Psi(\eta)^T J \dot{\Psi} (\eta) + \Psi(\eta) \mathrm{sk}(\Psi(\eta) \dot{\eta}) J \Psi(\eta).
\end{align}}
\subsection{Exponential control barrier functions(ECBF)}
\label{sec:ECBF}
% \ref{sec:CBF}節で定義されたように,
CBF can be applied when the derivative of $ h(x) $ is one.
We relax this relative degree condition and assume that $ h(x) $ has a higher relative degree $ r \geq 1 $, i.e., it satisfies the following equation,
\begin{align}
    \label{eq:h_relative_degree}
    h^{(r)}(x,u) = L_f^r h(x) + L_g L_f^{r-1} h(x)u.
\end{align}
Here, $L_g L_f^{r-1} h(x) \neq 0 $ and $ L_g L_f^2 h(x) = \cdots ~ L_g L_f^{r-2} h(x) = 0 $, $ \forall x \in D $
% We further define the following,
% \begin{align}
%     \label{eq:definition_of_eta_b}
%     \eta_b(x) \coloneqq
%     \begin{bmatrix}
%         h(x) \\
%         \dot{h}(x) \\
%         \ddot{h}(x) \\
%         \vdots \\
%         h^{(r-1)}(x)
%     \end{bmatrix}
%     =
%     \begin{bmatrix}
%         h(x) \\
%         L_f h(x) \\
%         L_f^2 h(x) \\
%         \vdots \\
%         L_f^{r-1} h(x)
%     \end{bmatrix}.
% \end{align}
% We assume that there exists a control input $ u \in U_{\mu} \subset \mathbb{R} $ for $ \mu \in \mathbb{R} $ such that $ L_f^r h(x) + L_g L_f^{r-1} h(x) u = \mu $.
% Then, the dynamics of $ h(x) $ can be described as the following linear system,
% \begin{align}
%     \label{eq:linear_system_of_h}
%     \dot{\eta}_b(x) &= F \eta_b(x) + G \mu, \\
%     h(x) &= C \eta_b (x)
% \end{align}
% where
% \begin{align}
%     \label{eq:definition_of_F_G_C}
%     F &= 
%     \begin{bmatrix}
%         0 & 1 & 0 & \cdots & 0 \\
%         0 & 0 & 1 & \cdots & 0 \\
%         \vdots & \vdots & \vdots & \ddots & \vdots \\
%         0 & 0 & 0 & \cdots & 1 \\
%         0 & 0 & 0 & \cdots & 0
%     \end{bmatrix},
%     G = 
%     \begin{bmatrix}
%         0 \\
%         0 \\
%         \vdots \\
%         0 \\
%         1   
%     \end{bmatrix}, \\
%     C &= [1 ~ 0 ~ \cdots ~ 0].
% \end{align}
% If we choose the state feedback $ \mu = - K_{\alpha} \eta_b (x) $, then $ h(x(t)) = C e^{(F - GK_{\alpha})t} \eta_b (x_0) $.
% Furthermore, if $ \mu \geq -K_{\alpha} \eta_b (x) $, then $ h(x(t)) \geq C e^{(F - GK_{\alpha})t} \eta_b (x_0) $ by comparison lemma.

\begin{definition}
    \cite{amesControlBarrierFunctions2019a}
    Given a set $ C \subset D \subset \mathbb{R} $, a function $ h : D \rightarrow \mathbb{R} $ that is $ r $ times continuously differentiable is an exponential control barrier function(ECBF) if there exists a column vector $ K_{\alpha} \in \mathbb{R}^r $ that satisfies the following equation for all $ x \in D $,
    \begin{align}
        \label{eq:definition_of_ECBF}
        \sup_{u \in U} [ L_f^r h(x) + L_g L_f^{r-1} h(x) u] \geq - K_{\alpha} \eta_b (x).
    \end{align}
    Here, $ \forall x \in \mathrm{Int} (C) $, $ h(x(t)) \geq C e^{(F - GK_{\alpha})t} \eta_b (x_0) $ whenever $ h(x_0) \geq 0 $.
    %何か表現がおかしい
\end{definition}

\section{Control design}
\label{sec:control_design}

\textcolor{red}{In this section, we design the control law of trajectory tracking for the two-wheeled drone.}
\begin{figure}[t]
\centering
\includegraphics[width=1\linewidth]{./drawing/pdf/control_structure_based_on_dynamics.pdf}
\caption{Block diagram of the proposed control structure.}
\label{fig:block_diagram_of_the_proposed_control_structure}
\end{figure}
The control structure of this paper is shown in Fig.~\ref{fig:block_diagram_of_the_proposed_control_structure}.
We design the control law of translational motion based on kinematics. 
Therefore, we design the velocity feedback control law for velocity projection, and by using this, we finally get the thrust.
Furthermore, we design the control law of rotational motion based on linearized dynamics.
\subsection{Velocity input design}
\label{subsec:velocity_input_design}

In this paper, we propose a trajectory tracking control law for two-wheeled drones based on the control law for unicycle mobile robots \cite{rodriguez-cortesNewGeometricTrajectory2022}.
Suppose that the position $ y $, $ z $ of the drone is $ X = [y ~ z]^T \in \mathbb{R}^2 $, the target position of $ y $, $ z $ is $ X_d = [y_d ~ z_d]^T \in \mathbb{R}^2 $, the position error of $ y $, $ z $ is $ \tilde{X} = [y - y_d ~ z - z_d ]^T = X - X_d \in \mathbb{R}^2 $, $ v_3^b = v_{3d}^b $, and the positive definite matrix $ K_p > 0 $.
From \eqref{eq:kinematics_of_translation} and \eqref{eq:kinematic_of_rotation_with_constraint_simplified_ver}, the $ yz $ translation kinematics and roll kinemtics is given by the following equation, 
\begin{align}
    \label{eq:translation_kinematics_of_yz_without_second_term}
    \dot{p} = \frac{1}{\cos \gamma}
    \begin{bmatrix}
        \cos \beta & - \sin \beta \\
        \sin \beta & \cos \beta
    \end{bmatrix}
    \begin{bmatrix}
        0 \\
        v_{3d}^b
    \end{bmatrix}
\end{align}

\begin{align}
    \label{eq:kinematics_of_roll}
    \dot{\beta} = r_{\beta}.
\end{align}
Considering the nonholonomic constraint $ v_2^b = 0 $, the desired velocity $ w $ is transformed into the projection velocity by the following equation,
\begin{align}
    \label{eq:velocity_projection}
    v_{3d}^b = w^T \left ( E^T R_2 \left (\beta \right ) E \right ) \left ( E^T e_3 \right )
\end{align}
where $ E = [e_2 ~ e_3] \in \mathbb{R}^{3 \times 2} $.
Then, the desired velocity $ w = [w_y ~ w_z] \in \mathbb{R}^2 $ is designed by the following translational controller,
\begin{align}
    \label{eq:controller_of_translation}
    w = \cos \gamma_d ( - K_p \tilde{X} + \dot{X}_d ).
\end{align}
Furthermore, the roll controller is given by the following equation,
\begin{align}
    \label{eq:roll_controller_for_kinematics}
    r_{\beta} &= r_{\beta_d} - k_R ( \mathrm{sk}(\tilde{R}))^{\vee}
\end{align}
where
\begin{align}
    \begin{cases}
    \tilde{R} &= R_d^T E^T R_2(\beta)E \\
    R_d &= \left [(-1)^{\wedge} \frac{w}{\|w\|} ~ \frac{w}{\|w\|} \right ] \\
    r_{\beta_d} &= (R_d^T \dot{R}_d)^{\vee}.
    \label{eq:definition_of_error_angular_velocity}
    \end{cases}
\end{align}
Therefore, the following theorem holds.
\begin{theorem}
    \label{theorem:kinematics_stability}
    Consider the two-wheeled drone kinematics \eqref{eq:translation_kinematics_of_yz_without_second_term}, \eqref{eq:kinematic_of_rotation_with_constraint_simplified_ver} in closed-loop with the controllers \eqref{eq:controller_of_translation}, \eqref{eq:velocity_projection}, and \eqref{eq:roll_controller_for_kinematics}, and assume $ v_3^b = v_{3d}^b $.
    Then, the origin of the system is almost global asymptotically stable.
\end{theorem}
\begin{proof}
    The proof is same as \cite{rodriguez-cortesNewGeometricTrajectory2022}.
\end{proof}
From here, we extend Theorem~\ref{theorem:kinematics_stability} to the entire system of two-wheeled drones.

\subsection{Linearlization of rotational dynamics and pitch controller}
\label{subsec:linearizer_and_pitch_controller}
We define a new virtual input $ \tilde{\tau}^b \in \mathbb{R}^3 $ as follows,
\begin{align}
    \label{eq:vertual_input_of_tau}
    \tau^b = J \Psi(\eta) \tilde{\tau}^b + \left (\Psi(\eta)^T \right )^{-1} C(\eta, \dot{\eta}) \dot{\eta} + \lambda_3 e_1.
\end{align}
Then, the rotational dynamics \eqref{eq:rotational_el_eq} can be linearized as follows,
\begin{align}
    \label{eq:linearized_dynamics_of_rotation}
    \ddot{\eta} = \tilde{\tau}^b.
\end{align}
When running on the wall, it is important for two-wheeled drones to push the drone against the wall.
This is because the stability of the drone against sideslip or side wind is increased and the grip is improved.
By \eqref{eq:linearized_dynamics_of_rotation}, the linearized dynamics of the pitch direction is given by the following equation,
\begin{align}
    \label{eq:linearized_dynamics_of_pitch}
    \ddot{\gamma} = \tilde{\tau}_{\gamma}^b.
\end{align}
We design the pitch controller as follows,
\begin{align}
    \label{eq:controller_of_pitch}
    \tilde{\tau}_{\gamma}^b = - k_{p_{\gamma}} (\gamma - \gamma_d) - k_{d_{\gamma}} (\dot{\gamma} - \dot{\gamma}_d)
\end{align}
where $ k_{p_{\gamma}}, k_{d_{\gamma}} > 0 $.

\subsection{Roll controller}
\label{subsec:controller_of_roll}
The linearized dynamics of the roll is given by the following equation,
\begin{align}
    \label{eq:roll_dynamics}
    \ddot{\beta} = \tilde{\tau}_{\beta}^b.
\end{align}
Then, the roll controller for \eqref{eq:roll_dynamics} is given by the following equation,
\begin{align}
        \tilde{\tau}_{\beta}^b &= - k_{\beta} (\dot{\beta} - r_{\beta}) + \dot{r}_{\beta}
        \label{eq:controller_of_roll}
\end{align}
where $ k_{\beta} > 0 $.

\subsection{Translational dynamics and velocity feedback}
\label{subsec:translational_dynamics_and_velocity_feedback}
%安定性の証明も書く
We design the controller to obtain the thrust from the  velocity projection $ v_{3d}^b $ obtained by the translational controller.
\eqref{eq:translational_dynamics} and \eqref{eq:kinematics_of_translation} give the following equation,
\begin{align*}
    f = m (f_1 \dot{v}_3^b + f_2 + f_3)
\end{align*}
where
\begin{align*}
    \begin{cases}
            f_1 &= 1 - \frac{r_{13} (r_{32} r_{21} - r_{22} r_{31})}{r_{11}(r_{32} r_{23} - r_{22} r_{33})}\\
            f_2 &= \frac{v_3^b \omega_2^b (r_{32} r_{21} - r_{22} r_{31}  - r_{11}) - g r_{11} r_{22} + \omega_2^b r_{13} v_1^b}{r_{11} (r_{32} r_{23} - r_{22} r_{33})}\\
            f_3 &= v_1^b \omega_2^b.
    \end{cases}
\end{align*}
Therefore, the velocity feedback control law $ \dot{v}_3^b = -k_3 (v_3^b - v_{3d}^b ) + \dot{v}_{3d}^b $ is applied by the positive constant $ k_3 > 0 $.
Then, the thrust is given by the following equation,
\begin{align*}
    f = m \left (f_1 \left ( -k_3 \left (v_3^b - v_{3d}^b \right ) + \dot{v}_{3d}^b \right ) + f_2 + f_3 \right ).
\end{align*}

Hence, the main theorem of the stability in this paper is obtained as follows.
\begin{theorem}
    Consider the closed-loop system of the kinematics \eqref{eq:kinematics_of_translation}, \eqref{eq:kinematic_of_rotation_with_constraint_simplified_ver}, and the dynamics \eqref{eq:translational_dynamics}, \eqref{eq:rotational_el_eq} of the two-wheeled drones with the controller \eqref{eq:controller_of_translation}, \eqref{eq:controller_of_pitch}, and \eqref{eq:controller_of_roll}.
    Then, the equilibrium points $ p_e = 0 $, $ \tilde{R} = I $, and $ \gamma_e = 0 $ are almost global asymptotically stable.
\end{theorem}
\begin{proof}
    From \eqref{eq:kinematics_of_translation}, the $ yz $ kinematics of the translational motion is given by the following equation,
    \begin{align}
        \dot{X} &= \frac{1}{\cos \gamma}(v_{3d}^b - \tilde{v}_{3}^b) (E^T R_2(\beta) E)(E^T e_3) \\
        &= \frac{v_{3d}^b}{\cos \gamma_d} (E^T R_2(\beta) E)(E^T e_3) + \Delta_2 + \Delta_3
        \label{eq:translation_kinematics_of_yz}
    \end{align}
    where $ \tilde{v}_{3}^b = v_{3d}^b - v_3^b $ and
    \begin{align}
        \label{eq:definition_of_Delta2}
        \Delta_2 = - \frac{v_{3d}^b}{\cos \gamma_d} (E^T R_2(\beta) E)(E^T e_3) \\
        + \frac{v_{3d}^b}{\cos \gamma} (E^T R_2(\beta) E)(E^T e_3)
    \end{align}
    \begin{align}
        \label{eq:definition_of_Delta3}
        \Delta_3 = - \frac{\tilde{v}_{3}^b}{\cos \gamma}  (E^T R_2(\beta) E)(E^T e_3).
    \end{align}
    % ロドリゲスさんと同じ式変形だから省略
    % Here, we consider the following equation without the term $ \Delta_2 $ and $ \Delta_3 $ in \eqref{eq:translation_kinematics_of_yz},
    % \begin{align}
    %     \label{eq:translation_kinematics_of_yz_without_second_term}
    %     \dot{X} = \frac{v_{3d}^b}{\cos \gamma_d} (E^T R_2(\beta) E)(E^T e_3).
    % \end{align}
    % Here, \eqref{eq:translation_kinematics_of_yz_without_second_term} can be rewritten as follows,
    % \begin{align}
    %     \label{eq:translation_dynamics_of_yz_without_second_term_transformation_2}
    %     \dot{X} = &\frac{1}{\cos \gamma_d} \frac{v_{3d}^b R_d (E^T e_3)}{(E^T e_3)^T \tilde{R} (E^T e_3)} + \frac{1}{\cos \gamma_d} \frac{v_{3d}^b}{(E^T e_3)^T \tilde{R} (E^T e_3)} \\
    %     &\left ( (E^T e_3)^T \tilde{R} (E^T e_3) (E^T R_2(\beta) E) - R_d(E^T e_3) \right ).
    % \end{align}
    % From \eqref{eq:velocity_projection} and the second equation of \eqref{eq:definition_of_error_angular_velocity}, we get the following equation,
    % \begin{align}
    %     \dot{\tilde{X}} &= \frac{1}{\cos \gamma_d} w + \frac{1}{\cos \gamma_d} \|w\| \\
    %     & \left ( \left ( \left (E^T e_3 \right )^T \tilde{R} \left (E^T e_3 \right ) \right ) R_d \tilde{R} \left (E^T e_3 \right ) - R_d \left (E^T e_3 \right ) \right ). \\
    %     &- \dot{X}_d.
    %     \label{eq:error_translation_dynamics_of_yz_without_second_term}
    % \end{align}
    % Therefore, the error system containing the translational kinematics, dynamics, velocity feedback, and the velocity projection was derived.
    % Then, \eqref{eq:error_translation_dynamics_of_yz_without_second_term} is given by the following equation,
    % \begin{align}
    %     \dot{\tilde{X}} &= - K_p \tilde{X} + \|w\| \\
    %     &\left (  \left ( \left ( E^T e_3 \right )^T \tilde{R} \left (E^T e_3 \right ) \right ) R_d \tilde{R} \left (E^T e_3 \right ) - R_d \left (E^T e_3\right ) \right ).
    %     \label{eq:error_translation_of_yz_without_second_term_with_controller}
    % \end{align}

    %以下は元論文と同じ部分,違う部分を分けるために式変形する部分
    From \eqref{eq:roll_dynamics} and \eqref{eq:controller_of_roll}, the linearized dynamics of the roll direction is given by the following equation,
    \begin{align}
        \label{eq:roll_dynamics_and_controller}
        \ddot{\beta} = - k_{\beta} (\dot{\beta} - r_{\beta}) + \dot{r}_{\beta}.
    \end{align}
    \eqref{eq:roll_dynamics_and_controller} can be rewritten as follows,
    \begin{align}
        \label{eq:error_dynamics_of_roll}
        \ddot{\beta}_e = -k_{\beta} \dot{\beta}_e
    \end{align}
    where $ \dot{\beta}_e = \dot{\beta} - r_{\beta} $.
    Therefore, the system \eqref{eq:error_dynamics_of_roll} is asymptotically stable.
    \eqref{eq:roll_dynamics_and_controller} can be rewritten as follows,
    \begin{align}
        \label{eq:roll_dynamics_and_controller_with_delta}
        \ddot{\beta} = \Delta_1 + \dot{r}_{\beta}
    \end{align}
    where $ \Delta_1 = -k_{\beta} (\dot{\beta} - r_{\beta}) $.
    Furthermore, by integrating both sides of \eqref{eq:roll_dynamics_and_controller_with_delta}, we get the following equation,
    \textcolor{red}{
    \begin{align}
        \label{eq:roll_dynamics_and_controller_with_delta_integrated}
        \dot{\beta} (t) = \int_{t_0}^t \ddot{\beta} (\tau) d\tau + \dot{\beta} (t_0) = \Delta_1' + r_{\beta}(t) - r_{\beta}(t_0) + \dot{\beta} (t_0)
    \end{align}
    where $ \Delta_1' = \int_{t_0}^t \Delta_1 d\tau $.
    By writing \eqref{eq:roll_dynamics_and_controller_with_delta_integrated} as $ SO(2) $, we get the following equation,
    \begin{align}
        \label{eq:roll_dynamics_and_controller_SO(2)}
        \dot{R} = R 
        \begin{bmatrix}
            0 & -\dot{\beta}(t)\\
            \dot{\beta}(t) & 0
        \end{bmatrix} = R
        \begin{bmatrix}
            0 & -r_{\beta}(t) \\
            r_{\beta}(t) & 0
        \end{bmatrix}
        + \Delta_1''
    \end{align}
    where
    \begin{align}
        \label{eq:definition_of_Delta1''}
        \Delta_1'' = R
        \begin{bmatrix}
            0 & -(\Delta_1' + r_{\beta}(t) - r_{\beta}(t_0) + \dot{\beta} (t_0))\\
            \Delta_1' + r_{\beta}(t) - r_{\beta}(t_0) + \dot{\beta} (t_0) & 0
        \end{bmatrix}.
    \end{align}
    Furthermore, \eqref{eq:roll_dynamics_and_controller_SO(2)} can be rewritten to the error system as follows,
    \begin{align}
        \label{eq:roll_dynamics_and_controller_SO(2)_error}
        \dot{\tilde{R}} = \tilde{R} \hat{\tilde{r}}_{\beta}(t) + \Delta_1'''
    \end{align}
    where $ \tilde{r}_{\beta} = r_{\beta} - r_{\beta_d} $.
    }
    From \eqref{eq:linearized_dynamics_of_pitch} and \eqref{eq:controller_of_pitch}, the linearized dynamics of the pitch direction is given by the following equation,
    \begin{align}
        \label{eq:pitch_dynamics_and_controller}
        \ddot{\gamma} - \ddot{\gamma}_d= - k_{p_{\gamma}} (\gamma - \gamma_d) - k_{d_{\gamma}} (\dot{\gamma} - \dot{\gamma}_d)
    \end{align}
    \eqref{eq:pitch_dynamics_and_controller} can be rewritten as follows,
    \begin{align}
        \label{eq:error_dynamics_of_pitch}
        \ddot{\gamma}_e = - k_{p_{\gamma}} \gamma_e - k_{d_{\gamma}} \dot{\gamma}_e
    \end{align}
    where $ \gamma_e = \gamma - \gamma_d $.
    Therefore, \eqref{eq:error_dynamics_of_pitch} is rewritten as follows,
    \begin{align}
        \label{eq:error_dynamics_of_pitch_matrix}
        \frac{d}{dt}
        \begin{bmatrix}
            \gamma_e \\
            \dot{\gamma}_e
        \end{bmatrix}
        =
        \begin{bmatrix}
            0 & 1 \\
            - k_{p_{\gamma}} & -k_{d_{\gamma}}
        \end{bmatrix}
        \begin{bmatrix}
            \gamma_e \\
            \dot{\gamma}_e
        \end{bmatrix}.
    \end{align}

    The closed-loop system of the two-wheeled drone is given by the following equation,
    \begin{align}
        \label{eq:stability_proof_of_cascade_1}
        \begin{cases}
            \dot{\tilde{X}} &= - K_p \tilde{X} + \| K_p \tilde{X} + \dot{X}_d \| \left ( \left ( \left (E^Te_3 \right )^T \tilde{R} \right ) R_d \tilde{R} (E^T e_3) - R_d (E^T e_3) \right ) + \Delta_2 + \Delta_3 \\
            \dot{\tilde{R}} &= - k_R \tilde{R} \mathrm{sk} (\tilde{R}) + \Delta_1''' \\
            \frac{d}{dt}
            \begin{bmatrix}
                \gamma_e \\
                \dot{\gamma}_e
            \end{bmatrix}
            &=
            \begin{bmatrix}
                0 & 1 \\
                - k_{p_{\gamma}} & -k_{d_{\gamma}}
            \end{bmatrix}
            \begin{bmatrix}
                \gamma_e \\
                \dot{\gamma}_e
            \end{bmatrix}.
        \end{cases}
    \end{align}
    From \eqref{eq:error_dynamics_of_pitch_matrix}, the equilibrium point $ \gamma_e = 0 $ is global asymptotically stable.
    From \eqref{eq:controller_of_roll}, $ \dot{\beta} $ converges to $ r_{\beta} $, so $ \Delta_1' $ converges to $ 0 $.
    Furthermore, $ \gamma $ converges to $ \gamma_d $, so $ \Delta_2 $ converges to $ 0 $.
    Furthermore, $ v_3^b $ converges to $ v_{3d}^b $, so $ \Delta_3 $ converges to $ 0 $.
    For these reasons, the equilibrium point $ p_e = 0 $, $ \tilde{R} = I $, and $ \gamma_e = 0 $ is almost global asymptotically stable. 
\end{proof}

\section{Control design using control barrier functions}
\label{sec:control_design_using_CBF}
\textcolor{red}{In this section, we design ECBF-QP to avoid obstacles considering sideslipping.}

\subsection{Problem settings and program formulation}
\label{subsec:problem_setting}
\begin{figure}[t]
    \centering
    \includegraphics[width=1\linewidth]{./drawing/pdf/CBF_image.pdf}
    \caption{Outline drawing of obstacle avoidance.}
    \label{fig:outline_drawing_of_obstacle_avoidance}
\end{figure}
Consider a two-wheeled drone moving on a wall with a circular obstacle of a center $ p_c = [0, y_c, z_c]^T $ and a radius $ r \in \mathbb{R} $.
We show the outline drawing of the control law of this paper in Fig.~\ref{fig:outline_drawing_of_obstacle_avoidance}.
The blue dotted line is the image of the target trajectory, and the green solid line is the image of the trajectory when the obstacle avoidance control is used.
When the target trajectory is inside the obstacle, it is necessary to avoid the obstacle.
However, if the drone tries to avoid the obstacle by rolling on the wall, the gravity component in the lateral direction becomes large, and there is a risk of slipping.
Therefore, the control objective of this paper is to avoid obstacles without slipping.
The condition for not slipping is expressed by the following equation,
\begin{align}
    \label{eq:condition_of_no_slip}
    \| \lambda_1 A^T \| \leq \mu \| \lambda_2 e_1 \|.
\end{align}
where coefficient of static friction is $ \mu \in \mathbb{R} $, slipping force is $ \| \lambda_1 A^T \| $, and pushing force to the wall is $ \| \lambda_2 e_1 \|$.
$ A^T = [0 ~ r_{22} ~ r_{32}]^T $ gives $ \| A^T \| = 1 $ and since $ \| \lambda_2 e_1 \| $ is pushing force, $ \lambda_2 \geq 0 $.
Therefore, we can transform the equation \eqref{eq:condition_of_no_slip} as follows,
\begin{align}
    \label{eq:condition_of_no_slip_2}
    \| \lambda_1 \| \leq  \mu \lambda_2.
\end{align}
Hence, the control objective of this paper is to design the input $ f $, $ \tau^b $ that satisfies the condition \eqref{eq:condition_of_no_slip_2} and the following equation,
\begin{align*}
    \begin{cases}
            \displaystyle \lim_{t \rightarrow \infty} 
            \left \| p - p_d \right \| = 0
        &~ {\mathrm{if} } ~ \| p_d - p_c \| > r  \\
        \| p - p_c \| \geq r &~ {\mathrm{otherwise} }.
    \end{cases}
\end{align*}
%\begin{remark}
%    \label{remark:1}
%    ここには注意を書いてください.
%    注意には番号がつくようになっています.
%\end{remark}

\begin{figure}[t]
    \centering
    \includegraphics[width=1\linewidth]{./drawing/pdf/control_structure_ECBF-QP_obst_sideslip_simple.pdf}
    \caption{Control structure of ECBF-QP.}
    \label{fig:control_structure_of_ECBF-QP}
\end{figure}
The control structure of the ECBF-QP combined with obstacle avoidance and sideslip is shown in Fig.~\ref{fig:control_structure_of_ECBF-QP}.
The constraint equation of ECBF-QP for obstacle avoidance contains $ \tilde{\tau}_{\beta}^b $, but doesn't contain $ \tilde{\tau}_{\gamma}^b $.
On the other hand, the constraint equation of ECBF-QP for sideslip contain both $ \tilde{\tau}_{\beta}^b $ and $ \tilde{\tau}_{\gamma}^b $.
Therefore, the linearized roll torque $ \tilde{\tau}_{\beta}^b $ is modified by the ECBF-QP for obstacle avoidance, and then, the linearized pitch torque $ \tilde{\tau}_{\gamma}^b $ is modified by the ECBF-QP for sideslip.

\subsection{Control design of ECBF-QP for obstacle avoidance}
\label{subsec:obstQP_roll_torque}
%ISCIE

We consider the control barrier function as follows,
\begin{align*}
    h_{\mathrm{obst}} = (y- y_c)^2 + (z- z_c)^2 - r^2.
\end{align*}
We consider the linearized roll torque $ \tilde{\tau}_{\beta}^b $ as an input.
% よって,並進方向の運動方程式より,$ \lambda_1 $, $ \lambda_2 $は$ \dot{\eta} $の次元に相当するため,入力$ \tilde{\tau}_{\beta}^b $は$ \dddot{p} $の次元であらわれることがわかる.
% したがって,$ h $の3階微分を考えると次式であらわせる.
% \begin{align}
%     \label{eq:dddh_obst_iscie}
%     \dddot{h} = 6 \ddot{y}\dot{y} - 2(z_c - z)\dddot{z} - 2(y_c - y)\dddot{y} + 6 \ddot{z} \dot{z}.
% \end{align}
% また,並進方向の運動方程式より,次式が得られる.
% \begin{align}
%     \label{eq:ddp_obst_iscie}
%     \ddot{p} = \frac{Rfe_3 - mge_3 - \lambda_1 A^T - \lambda_2 e_1}{m}.
% \end{align}
% よって,\eqref{eq:ddp_obst_iscie}を時間微分し,\eqref{eq:dddh_obst_iscie}の$ \dddot{y} $と$ \dddot{z} $に代入する.
Therefore, $ \dddot{h}_{\mathrm{obst}} + \alpha_1 h_{\mathrm{obst}} + \alpha_2 \dot{h}_{\mathrm{obst}} + \alpha_3 \ddot{h}_{\mathrm{obst}} \geq 0 $ can be written as follows,
\begin{align}
    \label{eq:obstQP}
    f_x + g_x \tilde{\tau}_{\beta}^b \leq 0
\end{align}
where $ f_x $, $ g_x $ can be obtained by  Symbolic Math Toolbox in MATLAB, especially $ g_x $ is as follows,
\begin{align*}
    g_x = 2 \left ( \left (z_c - z \right ) \sin \beta + \left (y_c - y \right ) \cos \beta \right ) \\
    \times (- \cos \beta \dot{z} + \sin \beta \dot{y}).
\end{align*}
Hence, ECBF-QP for obstacle avoidance is obtained as follows,
\begin{align}
    \label{eq:obstQP_final}
    \begin{cases}
        \tilde{\tau}_{\mathrm{obst}}^b = \argmin_{\tilde{\tau}_{\mathrm{obst}}^b \in \mathbb{R}} \| \tilde{\tau}_{\mathrm{obst}}^b - \tilde{\tau}_0^b \| ^2 \\
        \mathrm{s.t.} \quad f_x + g_x \tilde{\tau}_{\beta}^b \leq 0.
        \end{cases}
\end{align}
\subsection{Control design of ECBF-QP for sideslip}
\label{subsec:sideslip}
When avoiding obstacles, two-wheeled drones need to roll, so sideslip is likely to occur.
For HyTAQs, there is research on considering sideslipping during ground running \cite{wuUnifiedTerrestrialAerial2023}.
In this chapter, we design a control barrier function to avoid sideslip in addition to the control barrier function for obstacle avoidance and consider the linearized pitch toqrue $ \tilde{\tau}_{\gamma}^b $ as an input.

Condition not to sideslip can be written as follows,
\begin{align}
    \label{eq:slip_condition_general}
    f_{\mathrm{slip}} \leq \mu N
\end{align}
where $ f_{\mathrm{slip}} $ is the force to make the drone sideslip, $ N $ is the force to push the drone against the wall, and $ \mu $ is the coefficient of static friction.
The force to make the drone sideslip is $ \| \lambda_1 A^T \| $, and the force to push the drone against the wall is $ \| \lambda_2 e_1 \| $, so \eqref{eq:slip_condition_general} can be written as follows,
\begin{align}
    \label{eq:slip_condition_using_lambda}
    \| \lambda_1 A^T \| \leq \mu \| \lambda_2 e_1 \|.
\end{align}
$ \lambda_1 $ can take both positive and negative values, but $ \lambda_2 \geq 0 $, and $ \| A^T \| = 1 $ and $ \| e_1 \| = 1 $, so \eqref{eq:slip_condition_using_lambda} can be written as follows,
\begin{align}
    \label{eq:slip_condition_using_lambda_2}
    \| \lambda_1 \| \leq \mu \lambda_2.
\end{align}
The inequality \eqref{eq:slip_condition_using_lambda_2} is an inequality containing absolute values, so we consider the conditions by dividing the cases.

If $ \lambda_1 \geq 0 $, \eqref{eq:slip_condition_using_lambda_2} can be written as follows,
\begin{align}
    \label{eq:slip_condition_using_lambda_5}
    mg \sin \beta + f \mu \sin \gamma - m \cos \beta \dot{z} \dot{\beta} + m \sin \beta \dot{y} \dot{\beta} \geq 0.
\end{align}
Therefore, the control barrier function $ h_{\mathrm{slip}_1} $ can be defined as follows,
\begin{align}
    h_{\mathrm{slip}_1} \coloneqq mg \sin \beta + f \mu \sin \gamma \\
    - m \cos \beta \dot{z} \dot{\beta} + m \sin \beta \dot{y} \dot{\beta}.
    \label{eq:slip_control_barrier_function_1}
\end{align}
\eqref{eq:slip_control_barrier_function_1} is a function of $ \tilde{\tau}_{\gamma}^b $ in the second order, so the relative degree is 2.
Therefore, $ \ddot{h}_{\mathrm{slip}_1} + \alpha_4 \dot{h}_{\mathrm{slip}_1}  + \alpha_5 \dot{h}_{\mathrm{slip}_1} \geq 0 $ can be written as follows,
\begin{align}
    \label{eq:sideslipQP_1}
    f_x + g_x \tilde{\tau}_{\gamma}^b \leq 0
\end{align}
where $ f_x $, $ g_x $ can be obtained by Symbolic Math Toolbox in MATLAB, especially $ g_x $ is as follows,
\begin{align*}
    g_x = - \mu \cos \gamma f.
\end{align*}

If $ \lambda_1 < 0 $, the control barrier function $ h_{\mathrm{slip}_2} $ can be defined as follows similar to $ \lambda_1 \geq 0 $,
\begin{align}
    \label{eq:slip_control_barrier_function_2}
    h_{\mathrm{slip}_2} \coloneqq f \mu \sin \gamma - mg \sin \beta \\
    + m \cos \beta \dot{z} \dot{\beta} - m \sin \beta \dot{y} \dot{\beta}.  
\end{align}
Therefore, $ \ddot{h}_{\mathrm{slip}_2} + \alpha_4 \dot{h}_{\mathrm{slip}_2}  + \alpha_5 \dot{h}_{\mathrm{slip}_2} \geq 0 $ can be written as follows,
\begin{align}
    \label{eq:sideslipQP_2}
    f_x + g_x \tilde{\tau}_{\gamma}^b \leq 0
\end{align}
where $ f_x $, $ g_x $ can be obtained by Symbolic Math Toolbox in MATLAB, especially $ g_x $ is as follows,
\begin{align*}
    g_x = - \mu \cos \gamma f.
\end{align*}
Hence, ECBF-QP for sideslip is obtained as follows,
\begin{align}
    \label{eq:slipQP_final}
    \begin{cases}
        \tilde{\tau}_{\mathrm{slip}}^b = \argmin_{\tilde{\tau}_{\mathrm{slip}}^b \in \mathbb{R}} \| \tilde{\tau}_{\mathrm{slip}}^b - \tilde{\tau}_{\mathrm{obst}}^b \| ^2 \\
        \mathrm{s.t.} \quad f_x + g_x \tilde{\tau}_{\gamma}^b \leq 0.
        \end{cases}
\end{align}
