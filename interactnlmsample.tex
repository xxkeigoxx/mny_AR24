% interactnlmsample.tex
% v1.05 - August 2017

% For pLaTeX compile
\RequirePackage{plautopatch}
% Embedding fonts
\AtBeginDvi{\special{pdf:mapfile uptex-ipa.map}}

% \documentclass[]{interact}
\documentclass[dvipdfmx]{interact}

% Added by Nakano
\usepackage{graphicx}
\RequirePackage{mathtools}
\mathtoolsset{showonlyrefs}

% Loading packages similar to the lab report template
\usepackage{sty/ynlab}

\usepackage{epstopdf}% To incorporate .eps illustrations using PDFLaTeX, etc.
\usepackage[caption=false]{subfig}% Support for small, `sub' figures and tables
%\usepackage[nolists,tablesfirst]{endfloat}% To `separate' figures and tables from text if required
%\usepackage[doublespacing]{setspace}% To produce a `double spaced' document if required
%\setlength\parindent{24pt}% To increase paragraph indentation when line spacing is doubled

\usepackage[numbers,sort&compress]{natbib}% Citation support using natbib.sty
\bibpunct[, ]{[}{]}{,}{n}{,}{,}% Citation support using natbib.sty
\renewcommand\bibfont{\fontsize{10}{12}\selectfont}% Bibliography support using natbib.sty
\makeatletter% @ becomes a letter
\def\NAT@def@citea{\def\@citea{\NAT@separator}}% Suppress spaces between citations using natbib.sty
\makeatother% @ becomes a symbol again

\theoremstyle{plain}% Theorem-like structures provided by amsthm.sty
\newtheorem{theorem}{Theorem}[section]
\newtheorem{lemma}[theorem]{Lemma}
\newtheorem{corollary}[theorem]{Corollary}
\newtheorem{proposition}[theorem]{Proposition}

\theoremstyle{definition}
\newtheorem{definition}[theorem]{Definition}
\newtheorem{example}[theorem]{Example}

\theoremstyle{remark}
\newtheorem{remark}{Remark}
\newtheorem{notation}{Notation}

%コマンド追加
\newcommand{\argmax}{\mathop{\rm arg~max}\limits}
\newcommand{\argmin}{\mathop{\rm arg~min}\limits}

\begin{document}

% \articletype{ARTICLE TEMPLATE}% Specify the article type or omit as appropriate

\title{Obstacle Avoidance Control for Two-Wheeled Drones Considering Sideslip Based on Control Barrier Functions}

% \author{
% \name{A.~N. Author\textsuperscript{a}\thanks{CONTACT A.~N. Author. Email: latex.helpdesk@tandf.co.uk} and John Smith\textsuperscript{b}}
% \affil{\textsuperscript{a}Taylor \& Francis, 4 Park Square, Milton Park, Abingdon, UK; \textsuperscript{b}Institut f\"{u}r Informatik, Albert-Ludwigs-Universit\"{a}t, Freiburg, Germany}
% }
\author{
\name{Keigo Mori\textsuperscript{a}\thanks{CONTACT Keigo Mori. Email: k.mori.452@stn.nitech.ac.jp}, Satoshi Nakano\textsuperscript{a}, and Manabu Yamada\textsuperscript{a}}
\affil{\textsuperscript{a}Graduate School of Engineering, Nagoya Institute of Technology;
Gokiso-cho, Showa-ku, Nagoya, Aichi 466-8555, JAPAN}
}

\maketitle

\begin{abstract}
  % This template is for authors who are preparing a manuscript for a Taylor \& Francis journal using the \LaTeX\ document preparation system and the \texttt{interact} class file, which is available via selected journals' home pages on the Taylor \& Francis website.

In this paper, we propose an obstacle avoidance control for two-wheeled drones considering sideslip based on control barrier functions.
First, we derive the Lagrange equation of them on a wall.
Next, we propose a trajectory tracking control law for them and prove that the equilibrium point of the system is almost global asymptotically stable.
Then, we propose an obstacle avoidance control law for them considering sideslip on a wall.
Finally, We confirm the effectiveness of the proposed control law for them by numerical simulations.
\end{abstract}

\begin{keywords}
  Drone; HyTAQs; Nonholonomic System; Trajectory Tracking; Cascade System; Stability; ECBF-QP; Obstacle Avoidance; Sideslip
\end{keywords}

\section{Introduction}
Exterior wall tiles used in many buildings can be peeled off and falled because of aging.
In fact, it is mandatory to inspect them every 10 years.
% \cite{MLIT}.
Normally, the inspection is carried out by engineers, but the inspection depends on the experience of the engineers and involves work at high places.
Especially, among the labor accidents, the number of fatal accidents in construction industries is the highest, and the number of falling is the highest in construction industries.
% \cite{MHLW}.
To solve such problems, it is expected that drones will conduct hammering tests and infrared inspections without depending on experience and with less risk.
% \cite{dotaro2018drone, daisuke2016drone, miho2021drone}.
By manually controlling wheeled drones, engineers can inspect without taking risks.   
However, manual control depends on the experience of engineers.

Therefore, in this paper, we propose an autonomous control of two-wheeled drones.
Research on autonomous control of drones without wheels \cite{kooijmanTrajectoryTrackingQuadrotors2019, leeControlComplexManeuvers2011a, leeGeometricTrackingControl2010} and research on autonomous control of HyTAQs (Hybrid Terrestrial and Aerial Quadrotors) \cite{fanAutonomousHybridGround2019, kalantariDesignExperimentalValidation2013, wuMotionPlanningHyTAQs2022} have been conducted.
% , nonami2017drone
A two-wheeled drone has a system with a nonholonomic constraint.
%  \cite{shima1997nonlinear}. 
It is known that there is no continuous static state feedback controller that makes the origin of the system of a unicycle mobile robot almost global asymptotically stable \cite{brockett1983asymptotic}.  
Furthermore, the attitude of the two-wheeled drone is a nonlinear configuration space with a unit circle, so there is no continuous static state feedback that makes the system global asymptotically stable \cite{sanjay2000topological}.
Controllers for the two-wheeled drone with nonholonomic constraint has been proposed \cite{rodriguez-cortesNewGeometricTrajectory2022}.
% {Time-State_Control_Form}.
The two-wheeled drone has an additional degree of freedom because it is necessary to consider the rotation of pitch direction in addition to the rotation of a roll direction on a wall.
Therefore, in this paper, we extend the attitude control, and propose the control law that includes dynamics not considered in \cite{rodriguez-cortesNewGeometricTrajectory2022}.
Furthermore, since a drone has a cascade structure, it is necessary to consider the interconnection term \cite{lee2013nonlinear} to discuss stability.
We prove that the origin of  the system of the two-wheeled drone with the control law in this paper is almost globally asymptotically stable \cite{angeli2001almost}.

In wall inspections, there is a risk that the drone will be damaged if it collides with windows, pipes, etc. when running on the wall, so the two-wheeled drone needs to avoid obstacles on a wall.
Control barrier functions (CBF) has been studied in recent years to ensure the safety of systems \cite{amesControlBarrierFunction2017, amesControlBarrierFunctions2019a, huangGuaranteedVehicleSafety2019, liSurveyControlLyapunov2023}.
For example, it has been applied to ACC (Adaptive Cruise Control) of automobiles \cite{amesControlBarrierFunction2014, xiaoControlBarrierFunctions2019} and obstacle avoidance for nonholonomic systems\cite{desaiCLFCBFBasedQuadratic2022, marleySynergisticControlBarrier2021}.
There are also studies on control laws using CBFs for drones to avoid obstacles \cite{khanBarrierFunctionsCascaded2020, wuSafetycriticalControlPlanar2016}.
In this paper, we use ECBF to avoid obstacles.
In obstacle avoidance of the two-wheeled drone on a wall, the roll is likely to become large and it is easy to slip.
On the other hand, by tilting the pitch and pressing the two-wheeled drone against a wall, it is less likely to slip.
Therefore, we propose a control law to avoid slipping by modifying the torque of the pitch using control barrier function.
Furthermore, we conduct numerical simulations using the proposed control law to verify the effectiveness.
% \section{Main result}
% % Main result \cite{kalabic_14acc,murray_book,Bhat2000}.

% 数学的準備や主結果などのメイン部分 \cite{latex}.
\section{Preliminaries}
\subsection{Two-wheeled drone model}
\begin{figure}[t]
    \centering
    \includegraphics[width=1\linewidth]{./drawing/pdf/Two-Wheeled_Drone.pdf}
    \caption{Two-wheeled drone moving on a wall.}
    \label{fig:two-wheeled_drone_moving_on_a_wall}
\end{figure}

Consider a two-wheeled drone moving on a wall illustrated in Fig.~\ref{fig:two-wheeled_drone_moving_on_a_wall}.
We choose an inertial frame $e_i \in \mathbb{R}^3$, $i \in \{1, 2, 3 \}$ and a body-fixed frame $\{b_x, b_y, b_z\}$.
The origin of the body-fixed frame is the center of mass of the drone.
The mass of the drone is $m \in \mathbb{R}$, acceleration of gravity is $g \in \mathbb{R}$, inertia tensor is $J \in \mathbb{R}^{3 \times 3}$, rotation matrix is $R \in SO(3)$, $ZXY$ Euler angles is $\eta = [\alpha ~ \beta ~ \gamma]^T \in \mathbb{R}^3$, body angular velocity is $\omega^b = [\omega_1^b ~ \omega_2^b ~ \omega_3^b]^T \in \mathbb{R}^3$, position in the world frame is $p = [x ~ y ~ z]^T \in \mathbb{R}^3$, body velocity is $v^b = [v_1^b ~ v_2^b ~ v_3^b]^T \in \mathbb{R}^ 3$, thrust is $F^b = [0 ~ 0 ~ f] \in \mathbb{R}^3$, body torque is $\tau^b \in \mathbb{R}^3$.

The kinematics of the translational motion is given by
\begin{align}
    \label{eq:kinematics_of_translation}
    \dot{p} &= Rv^b = 
    \begin{bmatrix}
        0\\
        - \frac{v_3^b \sin \beta}{\cos \gamma}\\
        \frac{v_3^b \cos \beta}{\cos \gamma}
    \end{bmatrix}
    = \frac{1}{\cos \gamma} R_2(\beta) (v_3^b e_3)
\end{align}
where
\begin{align}
    R &=
    \begin{bmatrix}
        r_{11} & r_{12} & r_{13} \\
        r_{21} & r_{22} & r_{23} \\
        r_{31} & r_{32} & r_{33}
    \end{bmatrix}\\
    &=
    \begin{bmatrix}
        \cos \gamma & 0 & \sin \gamma \\
        \sin \beta \sin \gamma & \cos \beta & - \sin \beta \cos \gamma \\
        - \cos \beta \sin \gamma & \sin \beta & \cos \beta \cos \gamma
    \end{bmatrix}
    \label{eq:rotation_matrix}
\end{align}

\begin{align}
    \label{eq:definition_of_R_2}
    R_2(\beta) =
    \begin{bmatrix}
        1 & 0 & 0 \\
        0 & \cos \beta & - \sin \beta \\
        0 & \sin \beta & \cos \beta
    \end{bmatrix}.
\end{align}

The kinematics of the rotational motion in the $ZXY$ Euler angles representation is given by
\begin{align}
    \label{eq:kinematic_of_rotation_with_constraint_simplified_ver}
\dot{\eta} = \Phi(\eta) \omega^b
\end{align}
where
\begin{align*}
    \Phi(\eta) = 
    \begin{bmatrix}
        \frac{\sin \gamma}{\cos \beta} & 0 & \frac{\cos \gamma}{\cos \beta}\\
        \cos \gamma & 0 & - \sin \gamma\\
        \frac{\sin \beta \sin \gamma}{\cos \beta} & 1 & \frac{\sin \beta \cos \gamma}{\cos \beta}
    \end{bmatrix}.
\end{align*}
The inverse of $ \Phi(\eta) $ is $ \Psi = \Phi^{-1} $, which is given by
\begin{align}
    \label{eq:definition_of_Psi}
    \Psi(\eta) =
    \begin{bmatrix}
        \cos \beta \sin \gamma & \cos \gamma & 0 \\
        - \sin \beta & 0 & 1 \\
        \cos \beta \cos \gamma & - \sin \gamma & 0
    \end{bmatrix}.
\end{align}

The dynamics of the translational motion is given by
\begin{align}
	\label{eq:translational_dynamics}
	m\ddot{p}+mge_3 +\lambda_1 A^T +\lambda_2 e_1 = Rfe_3.
\end{align}

The dynamics of the rotational motion \cite{nonami_autonomous_2010} is given by
\begin{align}
    \label{eq:rotational_el_eq}
    M(\eta) \ddot{\eta} + C(\eta, \dot{\eta}) \dot{\eta} = \Psi(\eta)^T \tau^b - \lambda_3 e_1.
\end{align}

\subsection{Exponential control barrier functions(ECBF)}
\label{sec:ECBF}
% \ref{sec:CBF}節で定義されたように,
CBF can be applied when the derivative of $ h(x) $ is one.
We relax this relative degree condition and assume that $ h(x) $ has a higher relative degree $ r \geq 1 $, i.e., it satisfies the following equation,
\begin{align}
    \label{eq:h_relative_degree}
    h^{(r)}(x,u) = L_f^r h(x) + L_g L_f^{r-1} h(x)u.
\end{align}
Here, $L_g L_f^{r-1} h(x) \neq 0 $ and $ L_g L_f^2 h(x) = \cdots ~ L_g L_f^{r-2} h(x) = 0 $, $ \forall x \in D $
We further define the following,
\begin{align}
    \label{eq:definition_of_eta_b}
    \eta_b(x) \coloneqq
    \begin{bmatrix}
        h(x) \\
        \dot{h}(x) \\
        \ddot{h}(x) \\
        \vdots \\
        h^{(r-1)}(x)
    \end{bmatrix}
    =
    \begin{bmatrix}
        h(x) \\
        L_f h(x) \\
        L_f^2 h(x) \\
        \vdots \\
        L_f^{r-1} h(x)
    \end{bmatrix}.
\end{align}
We assume that there exists a control input $ u \in U_{\mu} \subset \mathbb{R} $ for $ \mu \in \mathbb{R} $ such that $ L_f^r h(x) + L_g L_f^{r-1} h(x) u = \mu $.
Then, the dynamics of $ h(x) $ can be described as the following linear system,
\begin{align}
    \label{eq:linear_system_of_h}
    \dot{\eta}_b(x) &= F \eta_b(x) + G \mu, \\
    h(x) &= C \eta_b (x)
\end{align}
where
\begin{align}
    \label{eq:definition_of_F_G_C}
    F &= 
    \begin{bmatrix}
        0 & 1 & 0 & \cdots & 0 \\
        0 & 0 & 1 & \cdots & 0 \\
        \vdots & \vdots & \vdots & \ddots & \vdots \\
        0 & 0 & 0 & \cdots & 1 \\
        0 & 0 & 0 & \cdots & 0
    \end{bmatrix},
    G = 
    \begin{bmatrix}
        0 \\
        0 \\
        \vdots \\
        0 \\
        1   
    \end{bmatrix}, \\
    C &= [1 ~ 0 ~ \cdots ~ 0].
\end{align}
If we choose the state feedback $ \mu = - K_{\alpha} \eta_b (x) $, then $ h(x(t)) = C e^{(F - GK_{\alpha})t} \eta_b (x_0) $.
Furthermore, if $ \mu \geq -K_{\alpha} \eta_b (x) $, then $ h(x(t)) \geq C e^{(F - GK_{\alpha})t} \eta_b (x_0) $ by comparison lemma.

\begin{definition}
    Given a set $ C \subset D \subset \mathbb{R} $, a function $ h : D \rightarrow \mathbb{R} $ that is $ r $ times continuously differentiable is an exponential control barrier function(ECBF) if there exists a column vector $ K_{\alpha} \in \mathbb{R}^r $ that satisfies the following equation for all $ x \in D $,
    \begin{align}
        \label{eq:definition_of_ECBF}
        \sup_{u \in U} [ L_f^r h(x) + L_g L_f^{r-1} h(x) u] \geq - K_{\alpha} \eta_b (x).
    \end{align}
    Here, $ \forall x \in \mathrm{Int} (C) $, $ h(x(t)) \geq C e^{(F - GK_{\alpha})t} \eta_b (x_0) $ whenever $ h(x_0) \geq 0 $.
    %何か表現がおかしい
\end{definition}

\section{Control design}
\label{sec:control_design}

\begin{figure}[t]
\centering
\includegraphics[width=1\linewidth]{./drawing/pdf/control_structure_based_on_dynamics.pdf}
\caption{Block diagram of the proposed control structure.}
\label{fig:block_diagram_of_the_proposed_control_structure}
\end{figure}
The control structure of this paper is shown in Fig.~\ref{fig:block_diagram_of_the_proposed_control_structure}.
We design the control law of translational motion based on kinematics. 
Therefore, we design the velocity feedback control law for velocity projection, and by using this, we finally get the thrust.
Furthermore, we design the control law of rotational motion based on linearized dynamics.
\subsection{Velocity input design}
\label{subsec:velocity_input_design}

In this paper, we propose a trajectory tracking control law for two-wheeled drones based on the control law for unicycle mobile robots \cite{rodriguez-cortesNewGeometricTrajectory2022}.
Suppose that the position $ y $, $ z $ of the drone is $ X = [y ~ z]^T \in \mathbb{R}^2 $, the target position of $ y $, $ z $ is $ X_d = [y_d ~ z_d]^T \in \mathbb{R}^2 $, the position error of $ y $, $ z $ is $ \tilde{X} = [y - y_d ~ z - z_d ]^T = X - X_d \in \mathbb{R}^2 $, $ v_3^b = v_{3d}^b $, and the positive definite matrix $ K_p > 0 $.
From \eqref{eq:kinematics_of_translation} and \eqref{eq:kinematic_of_rotation_with_constraint_simplified_ver}, the $ yz $ translation kinematics and roll kinemtics is given by the following equation, 
\begin{align}
    \label{eq:translation_kinematics_of_yz_without_second_term}
    \dot{p} = \frac{1}{\cos \gamma}
    \begin{bmatrix}
        \cos \beta & - \sin \beta \\
        \sin \beta & \cos \beta
    \end{bmatrix}
    \begin{bmatrix}
        0 \\
        v_{3d}^b
    \end{bmatrix}
\end{align}

\begin{align}
    \label{eq:kinematics_of_roll}
    \dot{\beta} = r_{\beta}.
\end{align}
Considering the nonholonomic constraint $ v_2^b = 0 $, the desired velocity $ w $ is transformed into the projection velocity by the following equation,
\begin{align}
    \label{eq:velocity_projection}
    v_{3d}^b = w^T \left ( E^T R_2 \left (\beta \right ) E \right ) \left ( E^T e_3 \right )
\end{align}
where $ E = [e_2 ~ e_3] \in \mathbb{R}^{3 \times 2} $.
Then, the desired velocity $ w = [w_y ~ w_z] \in \mathbb{R}^2 $ is designed by the following translational controller,
\begin{align}
    \label{eq:controller_of_translation}
    w = \cos \gamma_d ( - K_p \tilde{X} + \dot{X}_d ).
\end{align}
Furthermore, the roll controller is given by the following equation,
\begin{align}
    \label{eq:roll_controller_for_kinematics}
    r_{\beta} &= r_{\beta_d} - k_R ( \mathrm{sk}(\tilde{R}))^{\vee}
\end{align}
where
\begin{align}
    \begin{cases}
    \tilde{R} &= R_d^T E^T R_2(\beta)E \\
    R_d &= \left [(-1)^{\wedge} \frac{w}{\|w\|} ~ \frac{w}{\|w\|} \right ] \\
    r_{\beta_d} &= (R_d^T \dot{R}_d)^{\vee}.
    \label{eq:definition_of_error_angular_velocity}
    \end{cases}
\end{align}
Therefore, the following theorem holds.
\begin{theorem}
    \label{theorem:kinematics_stability}
    Consider the two-wheeled drone kinematics \eqref{eq:translation_kinematics_of_yz_without_second_term}, \eqref{eq:kinematic_of_rotation_with_constraint_simplified_ver} in closed-loop with the controllers \eqref{eq:controller_of_translation}, \eqref{eq:velocity_projection}, and \eqref{eq:roll_controller_for_kinematics}, and assume $ v_3^b = v_{3d}^b $.
    Then, the origin of the system is almost global asymptotically stable.
\end{theorem}
\begin{proof}
    The proof is same as \cite{rodriguez-cortesNewGeometricTrajectory2022}.
\end{proof}
From here, we extend Theorem~\ref{theorem:kinematics_stability} to the entire system of two-wheeled drones.

\subsection{Linearlization of rotational dynamics and pitch controller}
\label{subsec:linearizer_and_pitch_controller}
We define a new virtual input $ \tilde{\tau}^b \in \mathbb{R}^3 $ as follows,
\begin{align}
    \label{eq:vertual_input_of_tau}
    \tau^b = J \Psi(\eta) \tilde{\tau}^b + \left (\Psi(\eta)^T \right )^{-1} C(\eta, \dot{\eta}) \dot{\eta} + \lambda_3 e_1.
\end{align}
Then, the rotational dynamics \eqref{eq:rotational_el_eq} can be linearized as follows,
\begin{align}
    \label{eq:linearized_dynamics_of_rotation}
    \ddot{\eta} = \tilde{\tau}^b.
\end{align}
When running on the wall, it is important for two-wheeled drones to push the drone against the wall.
This is because the stability of the drone against sideslip or side wind is increased and the grip is improved.
By \eqref{eq:linearized_dynamics_of_rotation}, the linearized dynamics of the pitch direction is given by the following equation,
\begin{align}
    \label{eq:linearized_dynamics_of_pitch}
    \ddot{\gamma} = \tilde{\tau}_{\gamma}^b.
\end{align}
We design the pitch controller as follows,
\begin{align}
    \label{eq:controller_of_pitch}
    \tilde{\tau}_{\gamma}^b = - k_{p_{\gamma}} (\gamma - \gamma_d) - k_{d_{\gamma}} (\dot{\gamma} - \dot{\gamma}_d)
\end{align}
where $ k_{p_{\gamma}}, k_{d_{\gamma}} > 0 $.

\subsection{Roll controller}
\label{subsec:controller_of_roll}
The linearized dynamics of the roll is given by the following equation,
\begin{align}
    \label{eq:roll_dynamics}
    \ddot{\beta} = \tilde{\tau}_{\beta}^b.
\end{align}
Then, the roll controller for \eqref{eq:roll_dynamics} is given by the following equation,
\begin{align}
        \tilde{\tau}_{\beta}^b &= - k_{\beta} (\dot{\beta} - r_{\beta}) + \dot{r}_{\beta}
        \label{eq:controller_of_roll}
\end{align}
where $ k_{\beta} > 0 $.

\subsection{Translational dynamics and velocity feedback}
\label{subsec:translational_dynamics_and_velocity_feedback}
%安定性の証明も書く
We design the controller to obtain the thrust from the  velocity projection $ v_{3d}^b $ obtained by the translational controller.
\eqref{eq:translational_dynamics} and \eqref{eq:kinematics_of_translation} give the following equation,
\begin{align*}
    f = m (f_1 \dot{v}_3^b + f_2 + f_3)
\end{align*}
where
\begin{align*}
    \begin{cases}
            f_1 &= 1 - \frac{r_{13} (r_{32} r_{21} - r_{22} r_{31})}{r_{11}(r_{32} r_{23} - r_{22} r_{33})}\\
            f_2 &= \frac{v_3^b \omega_2^b (r_{32} r_{21} - r_{22} r_{31}  - r_{11}) - g r_{11} r_{22} + \omega_2^b r_{13} v_1^b}{r_{11} (r_{32} r_{23} - r_{22} r_{33})}\\
            f_3 &= v_1^b \omega_2^b.
    \end{cases}
\end{align*}
Therefore, the velocity feedback control law $ \dot{v}_3^b = -k_3 (v_3^b - v_{3d}^b ) + \dot{v}_{3d}^b $ is applied by the positive constant $ k_3 > 0 $.
Then, the thrust is given by the following equation,
\begin{align*}
    f = m \left (f_1 \left ( -k_3 \left (v_3^b - v_{3d}^b \right ) + \dot{v}_{3d}^b \right ) + f_2 + f_3 \right ).
\end{align*}

Hence, the main theorem of the stability in this paper is obtained as follows.
\begin{theorem}
    Consider the closed-loop system of the kinematics \eqref{eq:kinematics_of_translation}, \eqref{eq:kinematic_of_rotation_with_constraint_simplified_ver}, and the dynamics \eqref{eq:translational_dynamics}, \eqref{eq:rotational_el_eq} of the two-wheeled drones with the controller \eqref{eq:controller_of_translation}, \eqref{eq:controller_of_pitch}, and \eqref{eq:controller_of_roll}.
    Then, the equilibrium points $ p_e = 0 $, $ \tilde{R} = I $, and $ \gamma_e = 0 $ are almost global asymptotically stable.
\end{theorem}
\begin{proof}
    From \eqref{eq:kinematics_of_translation}, the $ yz $ kinematics of the translational motion is given by the following equation,
    \begin{align}
        \dot{X} &= \frac{1}{\cos \gamma}(v_{3d}^b - \tilde{v}_{3}^b) (E^T R_2(\beta) E)(E^T e_3) \\
        &= \frac{v_{3d}^b}{\cos \gamma_d} (E^T R_2(\beta) E)(E^T e_3) + \Delta_2 + \Delta_3
        \label{eq:translation_kinematics_of_yz}
    \end{align}
    where $ \tilde{v}_{3}^b = v_{3d}^b - v_3^b $ and
    \begin{align}
        \label{eq:definition_of_Delta2}
        \Delta_2 = - \frac{v_{3d}^b}{\cos \gamma_d} (E^T R_2(\beta) E)(E^T e_3) \\
        + \frac{v_{3d}^b}{\cos \gamma} (E^T R_2(\beta) E)(E^T e_3)
    \end{align}
    \begin{align}
        \label{eq:definition_of_Delta3}
        \Delta_3 = - \frac{\tilde{v}_{3}^b}{\cos \gamma}  (E^T R_2(\beta) E)(E^T e_3).
    \end{align}
    % ロドリゲスさんと同じ式変形だから省略
    % Here, we consider the following equation without the term $ \Delta_2 $ and $ \Delta_3 $ in \eqref{eq:translation_kinematics_of_yz},
    % \begin{align}
    %     \label{eq:translation_kinematics_of_yz_without_second_term}
    %     \dot{X} = \frac{v_{3d}^b}{\cos \gamma_d} (E^T R_2(\beta) E)(E^T e_3).
    % \end{align}
    % Here, \eqref{eq:translation_kinematics_of_yz_without_second_term} can be rewritten as follows,
    % \begin{align}
    %     \label{eq:translation_dynamics_of_yz_without_second_term_transformation_2}
    %     \dot{X} = &\frac{1}{\cos \gamma_d} \frac{v_{3d}^b R_d (E^T e_3)}{(E^T e_3)^T \tilde{R} (E^T e_3)} + \frac{1}{\cos \gamma_d} \frac{v_{3d}^b}{(E^T e_3)^T \tilde{R} (E^T e_3)} \\
    %     &\left ( (E^T e_3)^T \tilde{R} (E^T e_3) (E^T R_2(\beta) E) - R_d(E^T e_3) \right ).
    % \end{align}
    % From \eqref{eq:velocity_projection} and the second equation of \eqref{eq:definition_of_error_angular_velocity}, we get the following equation,
    % \begin{align}
    %     \dot{\tilde{X}} &= \frac{1}{\cos \gamma_d} w + \frac{1}{\cos \gamma_d} \|w\| \\
    %     & \left ( \left ( \left (E^T e_3 \right )^T \tilde{R} \left (E^T e_3 \right ) \right ) R_d \tilde{R} \left (E^T e_3 \right ) - R_d \left (E^T e_3 \right ) \right ). \\
    %     &- \dot{X}_d.
    %     \label{eq:error_translation_dynamics_of_yz_without_second_term}
    % \end{align}
    % Therefore, the error system containing the translational kinematics, dynamics, velocity feedback, and the velocity projection was derived.
    % Then, \eqref{eq:error_translation_dynamics_of_yz_without_second_term} is given by the following equation,
    % \begin{align}
    %     \dot{\tilde{X}} &= - K_p \tilde{X} + \|w\| \\
    %     &\left (  \left ( \left ( E^T e_3 \right )^T \tilde{R} \left (E^T e_3 \right ) \right ) R_d \tilde{R} \left (E^T e_3 \right ) - R_d \left (E^T e_3\right ) \right ).
    %     \label{eq:error_translation_of_yz_without_second_term_with_controller}
    % \end{align}

    %以下は元論文と同じ部分,違う部分を分けるために式変形する部分
    From \eqref{eq:roll_dynamics} and \eqref{eq:controller_of_roll}, the linearized dynamics of the roll direction is given by the following equation,
    \begin{align}
        \label{eq:roll_dynamics_and_controller}
        \ddot{\beta} = - k_{\beta} (\dot{\beta} - r_{\beta}) + \dot{r}_{\beta}.
    \end{align}
    \eqref{eq:roll_dynamics_and_controller} can be rewritten as follows,
    \begin{align}
        \label{eq:error_dynamics_of_roll}
        \ddot{\beta}_e = -k_{\beta} \dot{\beta}_e
    \end{align}
    where $ \dot{\beta}_e = \dot{\beta} - r_{\beta} $.
    Therefore, the system \eqref{eq:error_dynamics_of_roll} is asymptotically stable.
    \eqref{eq:roll_dynamics_and_controller} can be rewritten as follows,
    \begin{align}
        \label{eq:roll_dynamics_and_controller_with_delta}
        \ddot{\beta} = \Delta_1 + \dot{r}_{\beta}
    \end{align}
    where $ \Delta_1 = -k_{\beta} (\dot{\beta} - r_{\beta}) $.
    Furthermore, by integrating both sides of \eqref{eq:roll_dynamics_and_controller_with_delta}, we get the following equation,
    \begin{align}
    \label{eq:roll_dynamics_and_controller_with_delta_integrated}
        \dot{\beta} = \int \Delta_1 dt + r_{\beta} = \Delta_1' + r_{\beta}.
    \end{align}
    By writing \eqref{eq:roll_dynamics_and_controller_with_delta_integrated} as $ SO(2) $, we get the following equation,
    \begin{align}
        \label{eq:roll_dynamics_and_controller_SO(2)}
        \dot{R} = R \hat{r}_{\beta} + \Delta_1''.
    \end{align}
    Furthermore, \eqref{eq:roll_dynamics_and_controller_SO(2)} can be rewritten as follows,
    \begin{align}
        \label{eq:roll_dynamics_and_controller_SO(2)_error}
        \dot{\tilde{R}} = \tilde{R} \hat{\tilde{r}}_{\beta} + \Delta_1'''
    \end{align}
    where $ \tilde{r}_{\beta} = r_{\beta} - r_{\beta_d} $.
    
    From \eqref{eq:linearized_dynamics_of_pitch} and \eqref{eq:controller_of_pitch}, the linearized dynamics of the pitch direction is given by the following equation,
    \begin{align}
        \label{eq:pitch_dynamics_and_controller}
        \ddot{\gamma} - \ddot{\gamma}_d= - k_{p_{\gamma}} (\gamma - \gamma_d) - k_{d_{\gamma}} (\dot{\gamma} - \dot{\gamma}_d)
    \end{align}
    \eqref{eq:pitch_dynamics_and_controller} can be rewritten as follows,
    \begin{align}
        \label{eq:error_dynamics_of_pitch}
        \ddot{\gamma}_e = - k_{p_{\gamma}} \gamma_e - k_{d_{\gamma}} \dot{\gamma}_e
    \end{align}
    where $ \gamma_e = \gamma - \gamma_d $.
    Therefore, \eqref{eq:error_dynamics_of_pitch} is rewritten as follows,
    \begin{align}
        \label{eq:error_dynamics_of_pitch_matrix}
        \frac{d}{dt}
        \begin{bmatrix}
            \gamma_e \\
            \dot{\gamma}_e
        \end{bmatrix}
        =
        \begin{bmatrix}
            0 & 1 \\
            - k_{p_{\gamma}} & -k_{d_{\gamma}}
        \end{bmatrix}
        \begin{bmatrix}
            \gamma_e \\
            \dot{\gamma}_e
        \end{bmatrix}.
    \end{align}

    The closed-loop system of the two-wheeled drone is given by the following equation,
    \begin{align}
        \label{eq:stability_proof_of_cascade_1}
        \begin{cases}
            \dot{\tilde{X}} &= - K_p \tilde{X} + \| K_p \tilde{X} + \dot{X}_d \| \left ( \left ( \left (E^Te_3 \right )^T \tilde{R} \right ) R_d \tilde{R} (E^T e_3) - R_d (E^T e_3) \right ) + \Delta_2 + \Delta_3 \\
            \dot{\tilde{R}} &= - k_R \tilde{R} \mathrm{sk} (\tilde{R}) + \Delta_1''' \\
            \frac{d}{dt}
            \begin{bmatrix}
                \gamma_e \\
                \dot{\gamma}_e
            \end{bmatrix}
            &=
            \begin{bmatrix}
                0 & 1 \\
                - k_{p_{\gamma}} & -k_{d_{\gamma}}
            \end{bmatrix}
            \begin{bmatrix}
                \gamma_e \\
                \dot{\gamma}_e
            \end{bmatrix}.
        \end{cases}
    \end{align}
    From \eqref{eq:error_dynamics_of_pitch_matrix}, the equilibrium point $ \gamma_e = 0 $ is global asymptotically stable.
    From \eqref{eq:controller_of_roll}, $ \dot{\beta} $ converges to $ r_{\beta} $, so $ \Delta_1' $ converges to $ 0 $.
    Furthermore, $ \gamma $ converges to $ \gamma_d $, so $ \Delta_2 $ converges to $ 0 $.
    Furthermore, $ v_3^b $ converges to $ v_{3d}^b $, so $ \Delta_3 $ converges to $ 0 $.
    For these reasons, the equilibrium point $ p_e = 0 $, $ \tilde{R} = I $, and $ \gamma_e = 0 $ is almost global asymptotically stable. 
\end{proof}

\section{Control design using control barrier functions}
\label{sec:control_design_using_CBF}
In this chapter, we design a control barrier function (CBF) to avoid obstacles considering sideslipping.

\subsection{Problem settings and program formulation}
\label{subsec:problem_setting}
\begin{figure}[t]
    \centering
    \includegraphics[width=1\linewidth]{./drawing/pdf/CBF_image.pdf}
    \caption{Outline drawing of obstacle avoidance.}
    \label{fig:outline_drawing_of_obstacle_avoidance}
\end{figure}
Consider a two-wheeled drone moving on a wall with a circular obstacle of a center $ p_c = [0, y_c, z_c]^T $ and a radius $ r \in \mathbb{R} $.
We show the outline drawing of the control law of this paper in Fig.~\ref{fig:outline_drawing_of_obstacle_avoidance}.
The blue dotted line is the image of the target trajectory, and the green solid line is the image of the trajectory when the obstacle avoidance control is used.
When the target trajectory is inside the obstacle, it is necessary to avoid the obstacle.
However, if the drone tries to avoid the obstacle by rolling on the wall, the gravity component in the lateral direction becomes large, and there is a risk of slipping.
Therefore, the control objective of this paper is to avoid obstacles without slipping.
The condition for not slipping is expressed by the following equation,
\begin{align}
    \label{eq:condition_of_no_slip}
    \| \lambda_1 A^T \| \leq \mu \| \lambda_2 e_1 \|.
\end{align}
where coefficient of static friction is $ \mu \in \mathbb{R} $, slipping force is $ \| \lambda_1 A^T \| $, and pushing force to the wall is $ \| \lambda_2 e_1 \|$.
$ A^T = [0 ~ r_{22} ~ r_{32}]^T $ gives $ \| A^T \| = 1 $ and since $ \| \lambda_2 e_1 \| $ is pushing force, $ \lambda_2 \geq 0 $.
Therefore, we can transform the equation \eqref{eq:condition_of_no_slip} as follows,
\begin{align}
    \label{eq:condition_of_no_slip_2}
    \| \lambda_1 \| \leq  \mu \lambda_2.
\end{align}
Hence, the control objective of this paper is to design the input $ f $, $ \tau^b $ that satisfies the condition \eqref{eq:condition_of_no_slip_2} and the following equation,
\begin{align*}
    \begin{cases}
            \displaystyle \lim_{t \rightarrow \infty} 
            \left \| p - p_d \right \| = 0
        &~ {\mathrm{if} } ~ \| p_d - p_c \| > r  \\
        \| p - p_c \| \geq r &~ {\mathrm{otherwise} }.
    \end{cases}
\end{align*}
%\begin{remark}
%    \label{remark:1}
%    ここには注意を書いてください.
%    注意には番号がつくようになっています.
%\end{remark}

\begin{figure}[t]
    \centering
    \includegraphics[width=1\linewidth]{./drawing/pdf/control_structure_ECBF-QP_obst_sideslip_simple.pdf}
    \caption{Control structure of ECBF-QP.}
    \label{fig:control_structure_of_ECBF-QP}
\end{figure}
The control structure of the ECBF-QP combined with obstacle avoidance and sideslip is shown in Fig.~\ref{fig:control_structure_of_ECBF-QP}.
The constraint equation of ECBF-QP for obstacle avoidance contains $ \tilde{\tau}_{\beta}^b $, but doesn't contain $ \tilde{\tau}_{\gamma}^b $.
On the other hand, the constraint equation of ECBF-QP for sideslip contain both $ \tilde{\tau}_{\beta}^b $ and $ \tilde{\tau}_{\gamma}^b $.
Therefore, the linearized roll torque $ \tilde{\tau}_{\beta}^b $ is modified by the ECBF-QP for obstacle avoidance, and then, the linearized pitch torque $ \tilde{\tau}_{\gamma}^b $ is modified by the ECBF-QP for sideslip.

\subsection{Control design of ECBF-QP for obstacle avoidance}
\label{subsec:obstQP_roll_torque}
%ISCIE

We consider the control barrier function as follows,
\begin{align*}
    h_{\mathrm{obst}} = (y- y_c)^2 + (z- z_c)^2 - r^2.
\end{align*}
We consider the linearized roll torque $ \tilde{\tau}_{\beta}^b $ as an input.
% よって,並進方向の運動方程式より,$ \lambda_1 $, $ \lambda_2 $は$ \dot{\eta} $の次元に相当するため,入力$ \tilde{\tau}_{\beta}^b $は$ \dddot{p} $の次元であらわれることがわかる.
% したがって,$ h $の3階微分を考えると次式であらわせる.
% \begin{align}
%     \label{eq:dddh_obst_iscie}
%     \dddot{h} = 6 \ddot{y}\dot{y} - 2(z_c - z)\dddot{z} - 2(y_c - y)\dddot{y} + 6 \ddot{z} \dot{z}.
% \end{align}
% また,並進方向の運動方程式より,次式が得られる.
% \begin{align}
%     \label{eq:ddp_obst_iscie}
%     \ddot{p} = \frac{Rfe_3 - mge_3 - \lambda_1 A^T - \lambda_2 e_1}{m}.
% \end{align}
% よって,\eqref{eq:ddp_obst_iscie}を時間微分し,\eqref{eq:dddh_obst_iscie}の$ \dddot{y} $と$ \dddot{z} $に代入する.
Therefore, $ \dddot{h}_{\mathrm{obst}} + \alpha_1 h_{\mathrm{obst}} + \alpha_2 \dot{h}_{\mathrm{obst}} + \alpha_3 \ddot{h}_{\mathrm{obst}} \geq 0 $ can be written as follows,
\begin{align}
    \label{eq:obstQP}
    f_x + g_x \tilde{\tau}_{\beta}^b \leq 0
\end{align}
where $ f_x $, $ g_x $ can be obtained by  Symbolic Math Toolbox in MATLAB, especially $ g_x $ is as follows,
\begin{align*}
    g_x = 2 \left ( \left (z_c - z \right ) \sin \beta + \left (y_c - y \right ) \cos \beta \right ) \\
    \times (- \cos \beta \dot{z} + \sin \beta \dot{y}).
\end{align*}
Hence, ECBF-QP for obstacle avoidance is obtained as follows,
\begin{align}
    \label{eq:obstQP_final}
    \begin{cases}
        \tilde{\tau}_{\mathrm{obst}}^b = \argmin_{\tilde{\tau}_{\mathrm{obst}}^b \in \mathbb{R}} \| \tilde{\tau}_{\mathrm{obst}}^b - \tilde{\tau}_0^b \| ^2 \\
        \mathrm{s.t.} \quad f_x + g_x \tilde{\tau}_{\beta}^b \leq 0.
        \end{cases}
\end{align}
\subsection{Control design of ECBF-QP for sideslip}
\label{subsec:sideslip}
When avoiding obstacles, two-wheeled drones need to roll, so sideslip is likely to occur.
For HyTAQs, there is research on considering sideslipping during ground running \cite{wuUnifiedTerrestrialAerial2023}.
In this chapter, we design a control barrier function to avoid sideslip in addition to the control barrier function for obstacle avoidance and consider the linearized pitch toqrue $ \tilde{\tau}_{\gamma}^b $ as an input.

Condition not to sideslip can be written as follows,
\begin{align}
    \label{eq:slip_condition_general}
    f_{\mathrm{slip}} \leq \mu N
\end{align}
where $ f_{\mathrm{slip}} $ is the force to make the drone sideslip, $ N $ is the force to push the drone against the wall, and $ \mu $ is the coefficient of static friction.
The force to make the drone sideslip is $ \| \lambda_1 A^T \| $, and the force to push the drone against the wall is $ \| \lambda_2 e_1 \| $, so \eqref{eq:slip_condition_general} can be written as follows,
\begin{align}
    \label{eq:slip_condition_using_lambda}
    \| \lambda_1 A^T \| \leq \mu \| \lambda_2 e_1 \|.
\end{align}
$ \lambda_1 $ can take both positive and negative values, but $ \lambda_2 \geq 0 $, and $ \| A^T \| = 1 $ and $ \| e_1 \| = 1 $, so \eqref{eq:slip_condition_using_lambda} can be written as follows,
\begin{align}
    \label{eq:slip_condition_using_lambda_2}
    \| \lambda_1 \| \leq \mu \lambda_2.
\end{align}
The inequality \eqref{eq:slip_condition_using_lambda_2} is an inequality containing absolute values, so we consider the conditions by dividing the cases.

If $ \lambda_1 \geq 0 $, \eqref{eq:slip_condition_using_lambda_2} can be written as follows,
\begin{align}
    \label{eq:slip_condition_using_lambda_5}
    mg \sin \beta + f \mu \sin \gamma - m \cos \beta \dot{z} \dot{\beta} + m \sin \beta \dot{y} \dot{\beta} \geq 0.
\end{align}
Therefore, the control barrier function $ h_{\mathrm{slip}_1} $ can be defined as follows,
\begin{align}
    h_{\mathrm{slip}_1} \coloneqq mg \sin \beta + f \mu \sin \gamma \\
    - m \cos \beta \dot{z} \dot{\beta} + m \sin \beta \dot{y} \dot{\beta}.
    \label{eq:slip_control_barrier_function_1}
\end{align}
\eqref{eq:slip_control_barrier_function_1} is a function of $ \tilde{\tau}_{\gamma}^b $ in the second order, so the relative degree is 2.
Therefore, $ \ddot{h}_{\mathrm{slip}_1} + \alpha_4 \dot{h}_{\mathrm{slip}_1}  + \alpha_5 \dot{h}_{\mathrm{slip}_1} \geq 0 $ can be written as follows,
\begin{align}
    \label{eq:sideslipQP_1}
    f_x + g_x \tilde{\tau}_{\gamma}^b \leq 0
\end{align}
where $ f_x $, $ g_x $ can be obtained by Symbolic Math Toolbox in MATLAB, especially $ g_x $ is as follows,
\begin{align*}
    g_x = - \mu \cos \gamma f.
\end{align*}

If $ \lambda_1 < 0 $, the control barrier function $ h_{\mathrm{slip}_2} $ can be defined as follows similar to $ \lambda_1 \geq 0 $,
\begin{align}
    \label{eq:slip_control_barrier_function_2}
    h_{\mathrm{slip}_2} \coloneqq f \mu \sin \gamma - mg \sin \beta \\
    + m \cos \beta \dot{z} \dot{\beta} - m \sin \beta \dot{y} \dot{\beta}.  
\end{align}
Therefore, $ \ddot{h}_{\mathrm{slip}_2} + \alpha_4 \dot{h}_{\mathrm{slip}_2}  + \alpha_5 \dot{h}_{\mathrm{slip}_2} \geq 0 $ can be written as follows,
\begin{align}
    \label{eq:sideslipQP_2}
    f_x + g_x \tilde{\tau}_{\gamma}^b \leq 0
\end{align}
where $ f_x $, $ g_x $ can be obtained by Symbolic Math Toolbox in MATLAB, especially $ g_x $ is as follows,
\begin{align*}
    g_x = - \mu \cos \gamma f.
\end{align*}
Hence, ECBF-QP for sideslip is obtained as follows,
\begin{align}
    \label{eq:slipQP_final}
    \begin{cases}
        \tilde{\tau}_{\mathrm{slip}}^b = \argmin_{\tilde{\tau}_{\mathrm{slip}}^b \in \mathbb{R}} \| \tilde{\tau}_{\mathrm{slip}}^b - \tilde{\tau}_{\mathrm{obst}}^b \| ^2 \\
        \mathrm{s.t.} \quad f_x + g_x \tilde{\tau}_{\gamma}^b \leq 0.
        \end{cases}
\end{align}

\section{Simulation condition and results}
\label{chap:simulation}

In this chapter, we verify that the control objectives are achieved by simulation of the control design in this paper.
We consider $ m = 0.938 \si{kg} $, $ J=\mathrm{diag}[0.00933 ~ 0.00285 ~ 0.01130] \si{kg.m^2} $, $ g = 9.81 \si{m/s^2} $ as the physical parameters of the two-wheeled drone.
We also consider $ k_3 = 10 $ as the velocity feedback gain, $ K_p = \mathrm{diag}([0.5, 0.5 ,0.5]) $ as the position controller, $ k_R = 4 $, $ k_{\beta} = 2 $, $ k_{p_{\gamma}} = 5 $, $ k_{d_{\gamma}} = 5 $ as the attitude controller and $ [y_c ~ z_c] = [-0.2 ~ 1.0] \si{m} $, $ r = 0.5 \si{m} $, $ \alpha_1 = 0.4 $, $ \alpha_2 = 0.3 $, $ \alpha_3 = 0.3 $ as the parameters of the ECBF-QP for obstacle avoidance, and $ \mu = 0.5 $, $ \alpha_4 = 20 $, $ \alpha_5 = 10 $ as the parameters of the ECBF-QP for avoiding sideslip.
Furthermore, the initial values are $ p_0 = [0 ~ 0 ~ 0]^T \si{m} $, $ \dot{p}_0 = [0 ~ 0 ~ 0]^T \si{m/s} $, $ \eta_0 = [0 ~ 0 ~ \frac{5 \pi}{180}]^T \si{rad} $, $ \dot{\eta}_0 = [0 ~ 0 ~ 0]^T \si{rad/s} $, the target value is $ \gamma_d = \frac{15 \pi}{180} \si{rad} $, and the target trajectory on the wall $ p_d $ is given by the following equation,
\begin{align*}
      p_d =
      \begin{bmatrix}
        0\\
        0.2 \sin \frac{15 \pi}{180}t ~ ( 0 \leq t < 6) ,  \quad 0.2 ~ (6 \leq t)\\
        0.2t
      \end{bmatrix}.
\end{align*}

\begin{figure}[t]
\centering
\includegraphics[width=1\linewidth]{./simulation/AR/figs/with_obstQP_with_sideslipQP/pdf/Position.pdf}
\caption{Simulation result of position.}
\label{figs:simulation_result_of_position}
\end{figure}

\begin{figure}[t]
\centering
\includegraphics[width=1\linewidth]{./simulation/AR/figs/with_obstQP_with_sideslipQP/pdf/Attitude.pdf}
\caption{Simulation result of attitude.}
\label{figs:simulation_result_of_attitude}
\end{figure}
 
\begin{figure}[t]
\centering
\includegraphics[width=1\linewidth]{./simulation/AR/figs/with_obstQP_with_sideslipQP/pdf/Thrust.pdf}
\caption{Simulation result of thrust.}
\label{figs:simulation_result_of_thrust}
\end{figure}

\begin{figure}[t]
\centering
\includegraphics[width=1\linewidth]{./simulation/AR/figs/with_obstQP_with_sideslipQP/pdf/Torque.pdf}
\caption{Simulation result of torque.}
\label{figs:simulation_result_of_torque}
\end{figure}


\begin{figure}[t]
  \centering
  \includegraphics[width=1\linewidth]{./simulation/AR/figs/with_obstQP_with_sideslipQP/animation/pdf/trajectory.pdf}
  \caption{Animation of the simulation result.}
  \label{figs:animation_of_the_simulation_result}
\end{figure}

\begin{figure}[t]
  \centering
  \includegraphics[width=1\linewidth]{./simulation/AR/figs/with_obstQP_with_sideslipQP/pdf/Trajectory.pdf}
  \caption{Trajectory of the simulation result.}
  \label{figs:trajectory_of_the_simulation_result}
\end{figure}

\begin{figure}[t]
\centering
\includegraphics[width=1\linewidth]{./simulation/AR/figs/with_obstQP_with_sideslipQP/pdf/h_obst.pdf}
\caption{Obstacle CBF of the simulation result.}
\label{figs:obstacle_CBF_of_the_simulation_result}
\end{figure}

\begin{figure}[t]
\centering
\includegraphics[width=1\linewidth]{./simulation/AR/figs/with_obstQP_with_sideslipQP/pdf/h_slip.pdf}
\caption{Sideslip CBF of the simulation result.}
\label{figs:sideslip_CBF_of_the_simulation_result}
\end{figure}

We show the simulation results of the position, attitude, thrust, and torque in Fig.~\ref{figs:simulation_result_of_position}, Fig.~\ref{figs:simulation_result_of_attitude}, Fig.~\ref{figs:simulation_result_of_thrust}, Fig.~\ref{figs:simulation_result_of_torque}, respectively.
We also show the animation of the simulation result in Fig.~\ref{figs:animation_of_the_simulation_result} and the trajectory on the wall in Fig.~\ref{figs:trajectory_of_the_simulation_result}.
From Fig.~\ref{figs:simulation_result_of_position}, Fig.~\ref{figs:trajectory_of_the_simulation_result}.
We can see that the two-wheeled drone avoids the obstacle and follows the target trajectory.
Furthermore, we show the control barrier functions for obstacle avoidance and sideslip in Fig.~\ref{figs:obstacle_CBF_of_the_simulation_result}, Fig.~\ref{figs:sideslip_CBF_of_the_simulation_result}, respectively.
The control barrier function for sideslip is defined by the following equation,
\begin{align}
  h_{\mathrm{slip}} =
  \begin{cases}
     h_{\mathrm{slip}_1} &~ (\lambda_1 \geq 0), \\
     h_{\mathrm{slip}_2} &~ (\lambda_1 < 0).
    \label{eq:definition_of_h_slip}
  \end{cases}
\end{align}
From Fig.~\ref{figs:obstacle_CBF_of_the_simulation_result}, we can see that $ h_{\mathrm{obst}} \geq 0 $, and from Fig.~\ref{figs:sideslip_CBF_of_the_simulation_result}, we can see that $ h_{\mathrm{slip}} \geq 0 $.
Therefore, the two-wheeled drone satisfies the constraints of obstacle avoidance and sideslip.
% \section{Experiment.}
Experiment.
\section{Conclusion}
In this paper, we proposed an obstacle avoidance control for two-wheeled drones considering sideslip based on control barrier functions.
First, we derived the $ ZXY $ Euler angle-based Lagrange equation of them on a wall.

Next, we extended \cite{rodriguez-cortesNewGeometricTrajectory2022} and designed a control law for trajectory tracking of two-wheeled drones on a wall, and proved that the origin of the system is almost globally asymptotically stable.

Furthermore, we designed ECBF-QP for obstacle avoidance and sideslip using control barrier functions, and combined them.

Finally, we confirmed the effectiveness of the proposed control law for two-wheeled drones by numerical simulations.

As a future work, we will compare the $ SE(2) $-based control law for two-wheeled drones in \cite{morteza2017logarithmic} and the trajectory tracking control law in this paper, and consider the solvability conditions of CBF-QP.

% \section{\textcolor{red}{これ以降はテンプレートの文章.執筆前に必ず確認すること.}}

% \section{Introduction}

% In order to assist authors in the process of preparing a manuscript for a journal, the Taylor \& Francis `\textsf{Interact}' layout style has been implemented as a \LaTeXe\ class file based on the \texttt{article} document class. A \textsc{Bib}\TeX\ bibliography style file and a sample bibliography are also provided in order to assist with the formatting of your references.

% Commands that differ from or are provided in addition to standard \LaTeXe\ are described in this document, which is \emph{not} a substitute for a \LaTeXe\ tutorial.

% The \texttt{interactnlmsample.tex} file can be used as a template for a manuscript by cutting, pasting, inserting and deleting text as appropriate, using the preamble and the \LaTeX\ environments provided (e.g.\ \verb"\begin{abstract}", \verb"\begin{keywords}").


% \subsection{The \textsf{Interact} class file}\label{class}

% The \texttt{interact} class file preserves the standard \LaTeXe\ interface such that any document that can be produced using \texttt{article.cls} can also be produced with minimal alteration using the \texttt{interact} class file as described in this document.

% If your article is accepted for publication it will be typeset as the journal requires in Minion Pro and/or Myriad Pro. Since most authors will not have these fonts installed, the page make-up is liable to alter slightly with the change of font. Also, the \texttt{interact} class file produces only single-column format, which is preferred for peer review and will be converted to two-column format by the typesetter if necessary during preparation of the proofs. Please therefore do not try to match the typeset format exactly, but use the standard \LaTeX\ fonts instead and ignore details such as slightly long lines of text or figures/tables not appearing in exact synchronization with their citations in the text: these details will be dealt with by the typesetter. Similarly, it is unnecessary to spend time addressing warnings in the log file -- if your .tex file compiles to produce a PDF document that correctly shows how you wish your paper to appear, such warnings will not prevent your source files being imported into the typesetter's program.


% \subsection{Submission of manuscripts prepared using \emph{\LaTeX}}

% Manuscripts for possible publication should be submitted to the Editors for review as directed in the journal's Instructions for Authors, and in accordance with any technical instructions provided in the journal's ScholarOne Manuscripts or Editorial Manager site. Your \LaTeX\ source file(s), the class file and any graphics files will be required in addition to the final PDF version when final, revised versions of accepted manuscripts are submitted.

% Please ensure that any author-defined macros used in your article are gathered together in the preamble of your .tex file, i.e.\ before the \verb"\begin{document}" command. Note that if serious problems are encountered in the coding of a document (missing author-defined macros, for example), the typesetter may resort to rekeying it.


% \section{Using the \texttt{interact} class file}

% For convenience, simply copy the \texttt{interact.cls} file into the same directory as your manuscript files (you do not need to install it in your \TeX\ distribution). In order to use the \texttt{interact} document class, replace the command \verb"\documentclass{article}" at the beginning of your document with the command \verb"\documentclass{interact}".

% The following document-class options should \emph{not} be used with the \texttt{interact} class file:
% \begin{itemize}
%   \item \texttt{10pt}, \texttt{11pt}, \texttt{12pt} -- unavailable;
%   \item \texttt{oneside}, \texttt{twoside} -- not necessary, \texttt{oneside} is the default;
%   \item \texttt{leqno}, \texttt{titlepage} -- should not be used;
%   \item \texttt{twocolumn} -- should not be used (see Subsection~\ref{class});
%   \item \texttt{onecolumn} -- not necessary as it is the default style.
% \end{itemize}
% To prepare a manuscript for a journal that is printed in A4 (two column) format, use the \verb"largeformat" document-class option provided by \texttt{interact.cls}; otherwise the class file produces pages sized for B5 (single column) format by default. The \texttt{geometry} package should not be used to make any further adjustments to the page dimensions.

% %If your manuscript has supplementary content you can also use the \verb"interact" class file to prepare all or part of it using the \verb"suppldata" document-class option, which will suppress the `article history' date. This option \emph{must not} be used on any primary content. Note that authors are solely responsible for the preparation of all supplemental material.


% \section{Additional features of the \texttt{interact} class file}

% \subsection{Title, authors' names and affiliations, abstracts and article types}

% The title should be generated at the beginning of your article using the \verb"\maketitle" command.
% In the final version the author name(s) and affiliation(s) must be followed immediately by \verb"\maketitle" as shown below in order for them to be displayed in your PDF document.
% To prepare an anonymous version for double-blind peer review, you can put the \verb"\maketitle" between the \verb"\title" and the \verb"\author" in order to hide the author name(s) and affiliation(s) temporarily.
% Next you should include the abstract if your article has one, enclosed within an \texttt{abstract} environment.
% The \verb"\articletype" command is also provided as an \emph{optional} element which should \emph{only} be included if your article actually needs it.
% For example, the titles for this document begin as follows:
% \begin{verbatim}
% \articletype{ARTICLE TEMPLATE}

% \title{Taylor \& Francis \LaTeX\ template for authors (\textsf{Interact}
% layout + NLM reference style)}

% \author{
% \name{A.~N. Author\textsuperscript{a}\thanks{CONTACT A.~N. Author.
% Email: latex.helpdesk@tandf.co.uk} and John Smith\textsuperscript{b}}
% \affil{\textsuperscript{a}Taylor \& Francis, 4 Park Square, Milton
% Park, Abingdon, UK; \textsuperscript{b}Institut f\"{u}r Informatik,
% Albert-Ludwigs-Universit\"{a}t, Freiburg, Germany} }

% \maketitle

% \begin{abstract}
% This template is for authors who are preparing a manuscript for a
% Taylor \& Francis journal using the \LaTeX\ document preparation system
% and the \texttt{interact} class file, which is available via selected
% journals' home pages on the Taylor \& Francis website.
% \end{abstract}
% \end{verbatim}

% An additional abstract in another language (preceded by a translation of the article title) may be included within the \verb"abstract" environment if required.

% A graphical abstract may also be included if required. Within the \verb"abstract" environment you can include the code
% \begin{verbatim}
% \\\resizebox{25pc}{!}{\includegraphics{abstract.eps}}
% \end{verbatim}
% where the graphical abstract is to appear, where \verb"abstract.eps" is the name of the file containing the graphic (note that \verb"25pc" is the recommended maximum width, expressed in pica, for the graphical abstract in your manuscript).


% \subsection{Abbreviations}

% A list of abbreviations may be included if required, enclosed within an \texttt{abbreviations} environment, i.e.\ \verb"\begin{abbreviations}"\ldots\verb"\end{abbreviations}", immediately following the \verb"abstract" environment.


% \subsection{Keywords}

% A list of keywords may be included if required, enclosed within a \texttt{keywords} environment, i.e.\ \verb"\begin{keywords}"\ldots\verb"\end{keywords}". Additional keywords in other languages (preceded by a translation of the word `keywords') may also be included within the \verb"keywords" environment if required.


% \subsection{Subject classification codes}

% AMS, JEL or PACS classification codes may be included if required. The \texttt{interact} class file provides an \texttt{amscode} environment, i.e.\ \verb"\begin{amscode}"\ldots\verb"\end{amscode}", a \texttt{jelcode} environment, i.e.\ \verb"\begin{jelcode}"\ldots\verb"\end{jelcode}", and a \texttt{pacscode} environment, i.e.\ \verb"\begin{pacscode}"\ldots\verb"\end{pacscode}" to assist with this.


% \subsection{Additional footnotes to the title or authors' names}

% The \verb"\thanks" command may be used to create additional footnotes to the title or authors' names if required. Footnote symbols for this purpose should be used in the order
% $^\ast$~(coded as \verb"$^\ast$"), $\dagger$~(\verb"$\dagger$"), $\ddagger$~(\verb"$\ddagger$"), $\S$~(\verb"$\S$"), $\P$~(\verb"$\P$"), $\|$~(\verb"$\|$"),
% $\dagger\dagger$~(\verb"$\dagger\dagger$"), $\ddagger\ddagger$~(\verb"$\ddagger\ddagger$"), $\S\S$~(\verb"$\S\S$"), $\P\P$~(\verb"$\P\P$").

% Note that any \verb"footnote"s to the main text will automatically be assigned the superscript symbols 1, 2, 3, etc. by the class file.\footnote{If preferred, the \texttt{endnotes} package may be used to set the notes at the end of your text, before the bibliography. The symbols will be changed to match the style of the journal if necessary by the typesetter.}


% \section{Some guidelines for using the standard features of \LaTeX}

% \subsection{Sections}

% The \textsf{Interact} layout style allows for five levels of section heading, all of which are provided in the \texttt{interact} class file using the standard \LaTeX\ commands \verb"\section", \verb"\subsection", \verb"\subsubsection", \verb"\paragraph" and \verb"\subparagraph". Numbering will be automatically generated for all these headings by default.


% \subsection{Lists}

% Numbered lists are produced using the \texttt{enumerate} environment, which will number each list item with arabic numerals by default. For example,
% \begin{enumerate}
%   \item first item
%   \item second item
%   \item third item
% \end{enumerate}
% was produced by
% \begin{verbatim}
% \begin{enumerate}
%   \item first item
%   \item second item
%   \item third item
% \end{enumerate}
% \end{verbatim}
% Alternative numbering styles can be achieved by inserting an optional argument in square brackets to each \verb"item", e.g.\ \verb"\item[(i)] first item"\, to create a list numbered with roman numerals at level one.

% Bulleted lists are produced using the \texttt{itemize} environment. For example,
% \begin{itemize}
%   \item First bulleted item
%   \item Second bulleted item
%   \item Third bulleted item
% \end{itemize}
% was produced by
% \begin{verbatim}
% \begin{itemize}
%   \item First bulleted item
%   \item Second bulleted item
%   \item Third bulleted item
% \end{itemize}
% \end{verbatim}


% \subsection{Figures}

% The \texttt{interact} class file will deal with positioning your figures in the same way as standard \LaTeX. It should not normally be necessary to use the optional \texttt{[htb]} location specifiers of the \texttt{figure} environment in your manuscript; you may, however, find the \verb"[p]" placement option or the \verb"endfloat" package useful if a journal insists on the need to separate figures from the text.

% Figure captions appear below the figures themselves, therefore the \verb"\caption" command should appear after the body of the figure. For example, Figure~\ref{sample-figure} with caption and sub-captions is produced using the following commands:
% \begin{verbatim}
% \begin{figure}
% \centering
% \subfloat[An example of an individual figure sub-caption.]{%
% \resizebox*{5cm}{!}{\includegraphics{graph1.eps}}}\hspace{5pt}
% \subfloat[A slightly shorter sub-caption.]{%
% \resizebox*{5cm}{!}{\includegraphics{graph2.eps}}}
% \caption{Example of a two-part figure with individual sub-captions
%  showing that captions are flush left and justified if greater
%  than one line of text.} \label{sample-figure}
% \end{figure}
% \end{verbatim}
% \begin{figure}
%   \centering
%   \subfloat[An example of an individual figure sub-caption.]{%
%     \resizebox*{5cm}{!}{\includegraphics{figs/graph1.eps}}}\hspace{5pt}
%   \subfloat[A slightly shorter sub-caption.]{%
%     \resizebox*{5cm}{!}{\includegraphics{figs/graph2.eps}}}
%   \caption{Example of a two-part figure with individual sub-captions
%     showing that captions are flush left and justified if greater
%     than one line of text.} \label{sample-figure}
% \end{figure}

% To ensure that figures are correctly numbered automatically, the \verb"\label" command should be included just after the \verb"\caption" command, or in its argument.

% The \verb"\subfloat" command requires \verb"subfig.sty", which is called in the preamble of the \texttt{interactnlmsample.tex} file (to allow your choice of an alternative package if preferred) and included in the \textsf{Interact} \LaTeX\ bundle for convenience. Please supply any additional figure macros used with your article in the preamble of your .tex file.

% The source files of any figures will be required when the final, revised version of a manuscript is submitted. Authors should ensure that these are suitable (in terms of lettering size, etc.) for the reductions they envisage.

% The \texttt{epstopdf} package can be used to incorporate encapsulated PostScript (.eps) illustrations when using PDF\LaTeX, etc. Please provide the original .eps source files rather than the generated PDF images of those illustrations for production purposes.


% \subsection{Tables}

% The \texttt{interact} class file will deal with positioning your tables in the same way as standard \LaTeX. It should not normally be necessary to use the optional \texttt{[htb]} location specifiers of the \texttt{table} environment in your manuscript; you may, however, find the \verb"[p]" placement option or the \verb"endfloat" package useful if a journal insists on the need to separate tables from the text.

% The \texttt{tabular} environment can be used as shown to create tables with single horizontal rules at the head, foot and elsewhere as appropriate. The captions appear above the tables in the \textsf{Interact} style, therefore the \verb"\tbl" command should be used before the body of the table. For example, Table~\ref{sample-table} is produced using the following commands:
% \begin{table}
%   \tbl{Example of a table showing that its caption is as wide as
%     the table itself and justified.}
%   {\begin{tabular}{lcccccc} \toprule
%                                & \multicolumn{2}{l}{Type}                                   \\ \cmidrule{2-7}
%       Class                    & One                      & Two & Three & Four & Five & Six \\ \midrule
%       Alpha\textsuperscript{a} & A1                       & A2  & A3    & A4   & A5   & A6  \\
%       Beta                     & B2                       & B2  & B3    & B4   & B5   & B6  \\
%       Gamma                    & C2                       & C2  & C3    & C4   & C5   & C6  \\ \bottomrule
%     \end{tabular}}
%   \tabnote{\textsuperscript{a}This footnote shows how to include
%     footnotes to a table if required.}
%   \label{sample-table}
% \end{table}
% \begin{verbatim}
% \begin{table}
% \tbl{Example of a table showing that its caption is as wide as
%  the table itself and justified.}
% {\begin{tabular}{lcccccc} \toprule
%  & \multicolumn{2}{l}{Type} \\ \cmidrule{2-7}
%  Class & One & Two & Three & Four & Five & Six \\ \midrule
%  Alpha\textsuperscript{a} & A1 & A2 & A3 & A4 & A5 & A6 \\
%  Beta & B2 & B2 & B3 & B4 & B5 & B6 \\
%  Gamma & C2 & C2 & C3 & C4 & C5 & C6 \\ \bottomrule
% \end{tabular}}
% \tabnote{\textsuperscript{a}This footnote shows how to include
%  footnotes to a table if required.}
% \label{sample-table}
% \end{table}
% \end{verbatim}

% To ensure that tables are correctly numbered automatically, the \verb"\label" command should be included just before \verb"\end{table}".

% The \verb"\toprule", \verb"\midrule", \verb"\bottomrule" and \verb"\cmidrule" commands are those used by \verb"booktabs.sty", which is called by the \texttt{interact} class file and included in the \textsf{Interact} \LaTeX\ bundle for convenience. Tables produced using the standard commands of the \texttt{tabular} environment are also compatible with the \texttt{interact} class file.


% \subsection{Landscape pages}

% If a figure or table is too wide to fit the page it will need to be rotated, along with its caption, through 90$^{\circ}$ anticlockwise. Landscape figures and tables can be produced using the \verb"rotating" package, which is called by the \texttt{interact} class file. The following commands (for example) can be used to produce such pages.
% \begin{verbatim}
% \setcounter{figure}{1}
% \begin{sidewaysfigure}
% \centerline{\epsfbox{figname.eps}}
% \caption{Example landscape figure caption.}
% \label{landfig}
% \end{sidewaysfigure}
% \end{verbatim}
% \begin{verbatim}
% \setcounter{table}{1}
% \begin{sidewaystable}
%  \tbl{Example landscape table caption.}
%   {\begin{tabular}{@{}llllcll}
%     .
%     .
%     .
%   \end{tabular}}\label{landtab}
% \end{sidewaystable}
% \end{verbatim}
% Before any such float environment, use the \verb"\setcounter" command as above to fix the numbering of the caption (the value of the counter being the number given to the preceding figure or table). Subsequent captions will then be automatically renumbered accordingly. The \verb"\epsfbox" command requires \verb"epsfig.sty", which is called by the \texttt{interact} class file and is also included in the \textsf{Interact} \LaTeX\ bundle for convenience.

% Note that if the \verb"endfloat" package is used, one or both of the commands
% \begin{verbatim}
% \DeclareDelayedFloatFlavor{sidewaysfigure}{figure}
% \DeclareDelayedFloatFlavor{sidewaystable}{table}
% \end{verbatim}
% will need to be included in the preamble of your .tex file, after the \verb"endfloat" package is loaded, in order to process any landscape figures and/or tables correctly.


% \subsection{Theorem-like structures}

% A predefined \verb"proof" environment is provided by the \texttt{amsthm} package (which is called by the \texttt{interact} class file), as follows:
% \begin{proof}
%   More recent algorithms for solving the semidefinite programming relaxation are particularly efficient, because they explore the structure of the MAX-CUT problem.
% \end{proof}
% \noindent This was produced by simply typing:
% \begin{verbatim}
% \begin{proof}
% More recent algorithms for solving the semidefinite programming
% relaxation are particularly efficient, because they explore the
% structure of the MAX-CUT problem.
% \end{proof}
% \end{verbatim}
% Other theorem-like environments (theorem, definition, remark, etc.) need to be defined as required, e.g.\ using \verb"\newtheorem{theorem}{Theorem}" in the preamble of your .tex file (see the preamble of \verb"interactnlmsample.tex" for more examples). You can define the numbering scheme for these structures however suits your article best. Please note that the format of the text in these environments may be changed if necessary to match the style of individual journals by the typesetter during preparation of the proofs.


% \subsection{Mathematics}

% \subsubsection{Displayed mathematics}

% The \texttt{interact} class file will set displayed mathematical formulas centred on the page without equation numbers if you use the \texttt{displaymath} environment or the equivalent \verb"\[...\]" construction. For example, the equation
% \[
%   \hat{\theta}_{w_i} = \hat{\theta}(s(t,\mathcal{U}_{w_i}))
% \]
% was typeset using the commands
% \begin{verbatim}
% \[
%  \hat{\theta}_{w_i} = \hat{\theta}(s(t,\mathcal{U}_{w_i}))
% \]
% \end{verbatim}

% For those of your equations that you wish to be automatically numbered sequentially throughout the text for future reference, use the \texttt{equation} environment, e.g.
% \begin{equation}
%   \hat{\theta}_{w_i} = \hat{\theta}(s(t,\mathcal{U}_{w_i}))
% \end{equation}
% was typeset using the commands
% \begin{verbatim}
% \begin{equation}
%  \hat{\theta}_{w_i} = \hat{\theta}(s(t,\mathcal{U}_{w_i}))
% \end{equation}
% \end{verbatim}

% Part numbers for sets of equations may be generated using the \texttt{subequations} environment, e.g.
% \begin{subequations} \label{subeqnexample}
%   \begin{equation}
%     \varepsilon \rho w_{tt}(s,t) = N[w_{s}(s,t),w_{st}(s,t)]_{s},
%     \label{subeqnparta}
%   \end{equation}
%   \begin{equation}
%     w_{tt}(1,t)+N[w_{s}(1,t),w_{st}(1,t)] = 0,   \label{subeqnpartb}
%   \end{equation}
% \end{subequations}
% which was typeset using the commands
% \begin{verbatim}
% \begin{subequations} \label{subeqnexample}
% \begin{equation}
%      \varepsilon \rho w_{tt}(s,t) = N[w_{s}(s,t),w_{st}(s,t)]_{s},
%      \label{subeqnparta}
% \end{equation}
% \begin{equation}
%      w_{tt}(1,t)+N[w_{s}(1,t),w_{st}(1,t)] = 0,   \label{subeqnpartb}
% \end{equation}
% \end{subequations}
% \end{verbatim}
% This is made possible by the \texttt{amsmath} package, which is called by the class file. If you put a \verb"\label" just after the \verb"\begin{subequations}" command, references can be made to the collection of equations, i.e.\ `(\ref{subeqnexample})' in the example above. Or, as the example also shows, you can label and refer to each equation individually -- i.e.\ `(\ref{subeqnparta})' and `(\ref{subeqnpartb})'.

% Displayed mathematics should be given end-of-line punctuation appropriate to the running text sentence of which it forms a part, if required.

% \subsubsection{Math fonts}

% \paragraph{Superscripts and subscripts}
% Superscripts and subscripts will automatically come out in the correct size in a math environment (i.e.\ enclosed within \verb"\(...\)" or \verb"$...$" commands in running text, or within \verb"\[...\]" or the \texttt{equation} environment for displayed equations). Sub/superscripts that are physical variables should be italic, whereas those that are labels should be roman (e.g.\ $C_p$, $T_\mathrm{eff}$). If the subscripts or superscripts need to be other than italic, they must be coded individually.

% \paragraph{Upright Greek characters and the upright partial derivative sign}
% Upright lowercase Greek characters can be obtained by inserting the letter `u' in the control code for the character, e.g.\ \verb"\umu" and \verb"\upi" produce $\umu$ (used, for example, in the symbol for the unit microns -- $\umu\mathrm{m}$) and $\upi$ (the ratio of the circumference of a circle to its diameter). Similarly, the control code for the upright partial derivative $\upartial$ is \verb"\upartial". Bold lowercase as well as uppercase Greek characters can be obtained by \verb"{\bm \gamma}", for example, which gives ${\bm \gamma}$, and \verb"{\bm \Gamma}", which gives ${\bm \Gamma}$.


% \section*{Acknowledgement(s)}

% An unnumbered section, e.g.\ \verb"\section*{Acknowledgements}", may be used for thanks, etc.\ if required and included \emph{in the non-anonymous version} before any Notes or References.


% \section*{Disclosure statement}

% An unnumbered section, e.g.\ \verb"\section*{Disclosure statement}", may be used to declare any potential conflict of interest and included \emph{in the non-anonymous version} before any Notes or References, after any Acknowledgements and before any Funding information.


% \section*{Funding}

% An unnumbered section, e.g.\ \verb"\section*{Funding}", may be used for grant details, etc.\ if required and included \emph{in the non-anonymous version} before any Notes or References.


% \section*{Notes on contributor(s)}

% An unnumbered section, e.g.\ \verb"\section*{Notes on contributors}", may be included \emph{in the non-anonymous version} if required. A photograph may be added if requested.


% \section*{Nomenclature/Notation}

% An unnumbered section, e.g.\ \verb"\section*{Nomenclature}" (or \verb"\section*{Notation}"), may be included if required, before any Notes or References.


% \section*{Notes}

% An unnumbered `Notes' section may be included before the References (if using the \verb"endnotes" package, use the command \verb"\theendnotes" where the notes are to appear, instead of creating a \verb"\section*").


% \section{References}

% \subsection{References cited in the text}

% References should be cited in accordance with US National Library of Medicine (NLM) style. References are cited in the text by a number in square brackets (e.g. [1], [2,4,10], [11--15], \emph{not} [11]--[15]), in the order in which they first appear. For further details on this reference style, see the Instructions for Authors on the Taylor \& Francis website.

% Each bibliographic entry has a key, which is assigned by the author and is used to refer to that entry in the text. In this document, the key \verb"Jen05" in the citation form \verb"\cite{Jen05}" produces `\cite{Jen05}', and the keys \verb"{Sch02,Wen95}" in the citation form \verb"\cite{Sch02,Wen95}" produce `\cite{Jen05,Wen95}'. The citation for a range of bibliographic entries (e.g.\ `\cite{AG98,Men05,DCK03,Hor98,Ant03,Zha05,Rog05}') will automatically be produced by  \verb"\cite{Sha78,AG98,Smi75,Men05,DCK03,Hor98,Ant03,Zha05,Rog05,SRW05}". Optional notes may be included at the beginning and/or end of a citation by the use of square brackets, e.g.\ \verb"\cite[cf.][]{Gau05}" produces `\cite[cf.][]{Gau05}', \verb"\cite[p.356]{BGC04}" produces `\cite[p.356]{BGC04}', and \verb"\cite[see][p.73-–77]{PI51}" produces `\cite[see][p.73--77]{BGC04}'.


% \subsection{The list of references}

% References should be listed at the end of the main text in the order in which they are first cited in the text. The following list shows some sample references prepared in the Taylor \& Francis NLM style.

\bibliographystyle{sty/tfnlm}
% \bibliography{sty/interactnlmsample}
\bibliography{sty/reference}

% The commands
% \begin{verbatim}
% \usepackage[numbers,sort&compress]{natbib}
% \bibpunct[, ]{[}{]}{,}{n}{,}{,}
% \renewcommand\bibfont{\fontsize{10}{12}\selectfont}
% \makeatletter
% \def\NAT@def@citea{\def\@citea{\NAT@separator}}
% \makeatother
% \end{verbatim}
% need to be included in the preamble of your .tex file in order to generate the citations and bibliography as described above.

% Instead of typing the bibliography by hand, you may prefer to create the list of references using a \textsc{Bib}\TeX\ database. The \texttt{tfnlm.bst} file needs to be in your working folder or an appropriate directory, and the lines
% \begin{verbatim}
% \bibliographystyle{tfnlm}
% \bibliography{interactnlmsample}
% \end{verbatim}
% included where the list of references is to appear, where \texttt{tfnlm.bst} is the name of the \textsc{Bib}\TeX\ bibliography style file for Taylor \& Francis' NLM reference style and \texttt{interactnlmsample.bib} is the bibliographic database included with the \textsf{Interact}-NLM \LaTeX\ bundle (to be replaced with the name of your own .bib file). \LaTeX/\textsc{Bib}\TeX\ will extract from your .bib file only those references that are cited in your .tex file and list them in the References section.

% Please include a copy of your .bib file and/or the final generated .bbl file among your source files if your .tex file does not contain a reference list in a \texttt{thebibliography} environment.


% \section{Appendices}

% Any appendices should be placed after the list of references, beginning with the command \verb"\appendix" followed by the command \verb"\section" for each appendix title, e.g.
% \begin{verbatim}
% \appendix
% \section{This is the title of the first appendix}
% \section{This is the title of the second appendix}
% \end{verbatim}
% produces:\medskip

% \noindent\textbf{Appendix A. This is the title of the first appendix}\medskip

% \noindent\textbf{Appendix B. This is the title of the second appendix}\medskip

% \noindent Subsections, equations, figures, tables, etc.\ within appendices will then be automatically numbered as appropriate. Some theorem-like environments may need to have their counters reset manually (e.g.\ if they are not numbered within sections in the main text). You can achieve this by using \verb"\numberwithin{remark}{section}" (for example) just after the \verb"\appendix" command.

% Note that if the \verb"endfloat" package is used on a document containing any appendices, the \verb"\processdelayedfloats" command must be included immediately before the \verb"\appendix" command in order to ensure that the floats belonging to the main body of the text are numbered as such.

% %\processdelayedfloats %%% See above for an explanation of why this command might be needed here.

% \appendix

% \section{Troubleshooting}

% Authors may occasionally encounter problems with the preparation of a manuscript using \LaTeX. The appropriate action to take will depend on the nature of the problem:
% \begin{enumerate}
%   \item[(i)] If the problem is with \LaTeX\ itself, rather than with the actual macros, please consult an appropriate \LaTeXe\ manual for initial advice. If the solution cannot be found, or if you suspect that the problem does lie with the macros, then please contact Taylor \& Francis for assistance (\texttt{latex.helpdesk@tandf.co.uk}), clearly stating the title of the journal to which you are submitting.
%   \item[(ii)] Problems with page make-up (e.g.\ occasional overlong lines of text; figures or tables appearing out of order): please do not try to fix these using `hard' page make-up commands -- the typesetter will deal with such problems. (You may, if you wish, draw attention to particular problems when submitting the final version of your manuscript.)
%   \item[(iii)] If a required font is not available on your system, allow \TeX\ to substitute the font and specify which font is required in a covering letter accompanying your files.
% \end{enumerate}


% \section{Obtaining the template and class file}

% \subsection{Via the Taylor \& Francis website}

% This article template and the \texttt{interact} class file may be obtained via the `Instructions for Authors' pages of selected Taylor \& Francis journals.

% Please note that the class file calls up the open-source \LaTeX\ packages booktabs.sty, epsfig.sty and rotating.sty, which will, for convenience, unpack with the downloaded template and class file. The template calls for natbib.sty and subfig.sty, which are also supplied for convenience.


% \subsection{Via e-mail}

% This article template, the \texttt{interact} class file and the associated open-source \LaTeX\ packages are also available via e-mail. Requests should be addressed to \texttt{latex.helpdesk@tandf.co.uk}, clearly stating for which journal you require the template and class file.

\end{document}
