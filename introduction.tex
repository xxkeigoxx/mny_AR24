\section{Introduction}
Exterior wall tiles used in many buildings can be peeled off and falled because of aging.
In fact, it is mandatory to inspect them every 10 years.
% \cite{MLIT}.
Normally, the inspection is carried out by engineers, but the inspection depends on the experience of the engineers and involves work at high places.
Especially, among the labor accidents, the number of fatal accidents in construction industries is the highest, and the number of falling is the highest in construction industries.
% \cite{MHLW}.
To solve such problems, it is expected that drones will conduct hammering tests and infrared inspections without depending on experience and with less risk.
% \cite{dotaro2018drone, daisuke2016drone, miho2021drone}.
By manually controlling wheeled drones, engineers can inspect without taking risks.   
However, manual control depends on the experience of engineers.

Therefore, in this paper, we propose an autonomous control of two-wheeled drones.
Research on autonomous control of drones without wheels \cite{kooijmanTrajectoryTrackingQuadrotors2019, leeControlComplexManeuvers2011a, leeGeometricTrackingControl2010} and research on autonomous control of HyTAQs (Hybrid Terrestrial and Aerial Quadrotors) \cite{fanAutonomousHybridGround2019, kalantariDesignExperimentalValidation2013, wuMotionPlanningHyTAQs2022} have been conducted.
% , nonami2017drone
A two-wheeled drone has a system with a nonholonomic constraint.
%  \cite{shima1997nonlinear}. 
It is known that there is no continuous static state feedback controller that makes the origin of the system of a unicycle mobile robot almost global asymptotically stable \cite{brockett1983asymptotic}.  
Furthermore, the attitude of the two-wheeled drone is a nonlinear configuration space with a unit circle, so there is no continuous static state feedback that makes the system global asymptotically stable \cite{sanjay2000topological}.
Controllers for the two-wheeled drone with nonholonomic constraint has been proposed \cite{rodriguez-cortesNewGeometricTrajectory2022}.
% {Time-State_Control_Form}.
The two-wheeled drone has an additional degree of freedom because it is necessary to consider the rotation of pitch direction in addition to the rotation of a roll direction on a wall.
Therefore, in this paper, we extend the attitude control, and propose the control law that includes dynamics not considered in \cite{rodriguez-cortesNewGeometricTrajectory2022}.
Furthermore, since a drone has a cascade structure, it is necessary to consider the interconnection term \cite{lee2013nonlinear} to discuss stability.
We prove that the origin of  the system of the two-wheeled drone with the control law in this paper is almost globally asymptotically stable \cite{angeli2001almost}.

In wall inspections, there is a risk that the drone will be damaged if it collides with windows, pipes, etc. when running on the wall, so the two-wheeled drone needs to avoid obstacles on a wall.
Control barrier functions (CBF) has been studied in recent years to ensure the safety of systems \cite{amesControlBarrierFunction2017, amesControlBarrierFunctions2019a, huangGuaranteedVehicleSafety2019, liSurveyControlLyapunov2023}.
For example, it has been applied to ACC (Adaptive Cruise Control) of automobiles \cite{amesControlBarrierFunction2014, xiaoControlBarrierFunctions2019} and obstacle avoidance for nonholonomic systems\cite{desaiCLFCBFBasedQuadratic2022, marleySynergisticControlBarrier2021}.
There are also studies on control laws using CBFs for drones to avoid obstacles \cite{khanBarrierFunctionsCascaded2020, wuSafetycriticalControlPlanar2016}.
In this paper, we use ECBF to avoid obstacles.
In obstacle avoidance of the two-wheeled drone on a wall, the roll is likely to become large and it is easy to slip.
On the other hand, by tilting the pitch and pressing the two-wheeled drone against a wall, it is less likely to slip.
Therefore, we propose a control law to avoid slipping by modifying the torque of the pitch using control barrier function.
Furthermore, we conduct numerical simulations using the proposed control law to verify the effectiveness.